%\documentclass{article}
%\usepackage{beamerarticle}
\documentclass[serif,ignorenonframetext]{beamer}

% Macros for MATH 110 course dates

\newcommand{\commonTheme}{metropolis}
\newcommand{\commonColorTheme}{metropolis}

\newcommand{\commonAuthor}{Edward Doolittle}
\newcommand{\commonInstitute}{Department of Indigenous Knowledge and
  Science \\ First Nations University of Canada}
\newcommand{\commonCourse}{MATH 110 Calculus I}
\newcommand{\commonTerm}{202510}
\newcommand{\commonDate}{January 6, 2025}

% Review Material

% Lab 0
\newcommand{\commonEventNegativeOne}{LabNegativeOne}
\newcommand{\commonDateLabNegativeOne}{Monday, January 6, 2025}
\newcommand{\commonTitleLabNegativeOne}{MATH 110 Lab 0}
\newcommand{\commonSubtitleLabNegativeOne}{No Lab; Course Opens}

% Section 001
\newcommand{\commonEventZeroZeroOne}{ZeroZeroOne}
\newcommand{\commonDateZeroZeroOne}{Tuesday, January 7, 2025}
\newcommand{\commonTitleZeroZeroOne}{MATH 110 Review 0.1}
\newcommand{\commonSubtitleZeroZeroOne}{Review of Algebra}
\newcommand{\commonPSTitleZeroZeroOne}{MATH 110 Review Problem Set 0.1}

% Section 00A
\newcommand{\commonEventZeroZeroA}{ZeroZeroA}
\newcommand{\commonDateZeroZeroA}{Tuesday, January 7, 2025}
\newcommand{\commonTitleZeroZeroA}{MATH 110 Review 0.A}
\newcommand{\commonSubtitleZeroZeroA}{Review of Inequalities and
  Absolute Values}
\newcommand{\commonPSTitleZeroZeroA}{MATH 110 Review Problem Set 0.A}

% Section 00B
\newcommand{\commonEventZeroZeroB}{ZeroZeroB}
\newcommand{\commonDateZeroZeroB}{Tuesday, January 7, 2025}
\newcommand{\commonTitleZeroZeroB}{MATH 110 Review 0.B}
\newcommand{\commonSubtitleZeroZeroB}{Review of Coordinate Geometry
  and Lines}
\newcommand{\commonPSTitleZeroZeroB}{MATH 110 Review Problem Set 0.B}

% Section 00C
\newcommand{\commonEventZeroZeroC}{ZeroZeroC}
\newcommand{\commonDateZeroZeroC}{Thursday, January 9, 2025}
\newcommand{\commonTitleZeroZeroC}{MATH 110 Review 0.C}
\newcommand{\commonSubtitleZeroZeroC}{Review of Graphs of Second
  Degree Equations}
\newcommand{\commonPSTitleZeroZeroC}{MATH 110 Review Problem Set 0.C}

% Section 00D
\newcommand{\commonEventZeroZeroD}{ZeroZeroD}
\newcommand{\commonDateZeroZeroD}{Thursday, January 9, 2025}
\newcommand{\commonTitleZeroZeroD}{MATH 110 Review 0.D}
\newcommand{\commonSubtitleZeroZeroD}{Review of Trigonometry}
\newcommand{\commonPSTitleZeroZeroD}{MATH 110 Review Problem Set 0.D}

% Section 011
\newcommand{\commonEventZeroOneOne}{ZeroOneOne}
\newcommand{\commonDateZeroOneOne}{Thursday, January 9, 2025}
\newcommand{\commonTitleZeroOneOne}{MATH 110 Review 1.1}
\newcommand{\commonSubtitleZeroOneOne}{Review of Functions}
\newcommand{\commonPSTitleZeroOneOne}{MATH 110 Review Problem Set 1.1}


% Main Course

% Lab 1
\newcommand{\commonEventZero}{LabZero}
\newcommand{\commonDateLabZero}{Monday, January 13, 2025}
\newcommand{\commonTitleLabZero}{MATH 110 Lab 1}
\newcommand{\commonSubtitleLabZero}{Quiz 0: STACK, Onboarding}

% Section 1.4
\newcommand{\commonEventOne}{ZeroOneFour}
\newcommand{\commonDateZeroOneFour}{Tuesday, January 14, 2025}
\newcommand{\commonTitleZeroOneFour}{MATH 110 Lecture 1.4}
\newcommand{\commonSubtitleZeroOneFour}{The Tangent and Velocity Problems}
\newcommand{\commonPSTitleZeroOneFour}{MATH 110 Problem Set 1.4}

% Section 1.5
\newcommand{\commonEventTwo}{ZeroOneFive}
\newcommand{\commonDateZeroOneFive}{Thursday, January 16, 2025}
\newcommand{\commonTitleZeroOneFive}{MATH 110 Lecture 1.5}
\newcommand{\commonSubtitleZeroOneFive}{The Limit of a Function}
\newcommand{\commonPSTitleZeroOneFive}{MATH 110 Problem Set 1.5}

% Lab 2
\newcommand{\commonEventThree}{LabOne}
\newcommand{\commonDateLabOne}{Monday, January 20, 2025}
\newcommand{\commonTitleLabOne}{MATH 110 Lab 2}
\newcommand{\commonSubtitleLabOne}{Quiz 1: Review}

% Section 1.6
\newcommand{\commonEventFour}{ZeroOneSix}
\newcommand{\commonDateZeroOneSix}{Tuesday, January 21, 2025}
\newcommand{\commonTitleZeroOneSix}{MATH 110 Lecture 1.6}
\newcommand{\commonSubtitleZeroOneSix}{Calculating Limits Using the Limit Laws}
\newcommand{\commonPSTitleZeroOneSix}{MATH 110 Problem Set 1.6}

% Section 1.7
\newcommand{\commonEventFive}{ZeroOneSeven}
\newcommand{\commonDateZeroOneSeven}{(Not covered)}
\newcommand{\commonTitleZeroOneSeven}{MATH 110 Lecture 1.7}
\newcommand{\commonSubtitleZeroOneSeven}{The Precise Definition of a Limit}
\newcommand{\commonPSTitleZeroOneSeven}{MATH 110 Problem Set 1.7}

% Section 1.8
\newcommand{\commonEventSix}{ZeroOneEight}
\newcommand{\commonDateZeroOneEight}{Thursday, January 23, 2025}
\newcommand{\commonTitleZeroOneEight}{MATH 110 Lecture 1.8}
\newcommand{\commonSubtitleZeroOneEight}{Continuity}
\newcommand{\commonPSTitleZeroOneEight}{MATH 110 Problem Set 1.8}

% Lab 3
\newcommand{\commonEventSeven}{LabTwo}
\newcommand{\commonDateLabTwo}{Monday, January 27, 2025}
\newcommand{\commonTitleLabTwo}{MATH 110 Lab 3}
\newcommand{\commonSubtitleLabTwo}{Quiz 2: Sections 1.4, 1.5}

% Section 2.1
\newcommand{\commonEventEight}{ZeroTwoOne}
\newcommand{\commonDateZeroTwoOne}{Tuesday, January 28, 2025}
\newcommand{\commonTitleZeroTwoOne}{MATH 110 Lecture 2.1}
\newcommand{\commonSubtitleZeroTwoOne}{Derivatives and Rates of Change}
\newcommand{\commonPSTitleZeroTwoOne}{MATH 110 Problem Set 2.1}

% Section 2.2
\newcommand{\commonEventNine}{ZeroTwoTwo}
\newcommand{\commonDateZeroTwoTwo}{Thursday, January 30, 2025}
\newcommand{\commonTitleZeroTwoTwo}{MATH 110 Lecture 2.2}
\newcommand{\commonSubtitleZeroTwoTwo}{The Derivative as a Function}
\newcommand{\commonPSTitleZeroTwoTwo}{MATH 110 Problem Set 2.2}

% Lab 4
\newcommand{\commonEventTen}{LabThree}
\newcommand{\commonDateMTOne}{Monday, February 3, 2025} 
\newcommand{\commonDateLabThree}{Monday, February 3, 2025}
\newcommand{\commonTitleLabThree}{MATH 110 Lab 4}
\newcommand{\commonSubtitleLabThree}{Midterm: Review, Chapter 1}

% Section 2.3
\newcommand{\commonEventEleven}{ZeroTwoThree}
\newcommand{\commonDateZeroTwoThree}{Tuesday, February 4, 2025}
\newcommand{\commonTitleZeroTwoThree}{MATH 110 Lecture 2.3}
\newcommand{\commonSubtitleZeroTwoThree}{Differentiation Formulas}
\newcommand{\commonPSTitleZeroTwoThree}{MATH 110 Problem Set 2.3}

% Section 2.4
\newcommand{\commonEventTwelve}{ZeroTwoFour}
\newcommand{\commonDateZeroTwoFour}{Thursday, February 6, 2025}
\newcommand{\commonTitleZeroTwoFour}{MATH 110 Lecture 2.4}
\newcommand{\commonSubtitleZeroTwoFour}{Derivatives of Trigonometric Functions}
\newcommand{\commonPSTitleZeroTwoFour}{MATH 110 Problem Set 2.4}

% Lab 5
\newcommand{\commonEventThirteen}{LabFour}
\newcommand{\commonDateLabFour}{Monday, February 10, 2025}
\newcommand{\commonTitleLabFour}{MATH 110 Lab 5}
\newcommand{\commonSubtitleLabFour}{Quiz 3: Sections 2.1, 2.2}

% Section 2.5
\newcommand{\commonEventFourteen}{ZeroTwoFive}
\newcommand{\commonDateZeroTwoFive}{Tuesday, February 11, 2025}
\newcommand{\commonTitleZeroTwoFive}{MATH 110 Lecture 2.5}
\newcommand{\commonSubtitleZeroTwoFive}{The Chain Rule}
\newcommand{\commonPSTitleZeroTwoFive}{MATH 110 Problem Set 2.5}

% Section 2.6
\newcommand{\commonEventFifteen}{ZeroTwoSix}
\newcommand{\commonDateZeroTwoSix}{Thursday, February 13, 2025}
\newcommand{\commonTitleZeroTwoSix}{MATH 110 Lecture 2.6}
\newcommand{\commonSubtitleZeroTwoSix}{Implicit Differentiation}
\newcommand{\commonPSTitleZeroTwoSix}{MATH 110 Problem Set 2.6}

% Lab 6
\newcommand{\commonEventSixteen}{LabFive}
\newcommand{\commonDateLabFive}{Monday, February 24, 2025}
\newcommand{\commonTitleLabFive}{MATH 110 Lab 6}
\newcommand{\commonSubtitleLabFive}{Quiz 4: Sections 2.3, 2.4}

% Section 2.7
\newcommand{\commonEventSeventeen}{ZeroTwoSeven}
\newcommand{\commonDateZeroTwoSeven}{Tuesday, February 25, 2025}
\newcommand{\commonTitleZeroTwoSeven}{MATH 110 Lecture 2.7}
\newcommand{\commonSubtitleZeroTwoSeven}{Rates of Change in the
  Natural and Social Sciences}
\newcommand{\commonPSTitleZeroTwoSeven}{MATH 110 Problem Set 2.7}

% Section 2.8
\newcommand{\commonEventEighteen}{ZeroTwoEight}
\newcommand{\commonDateZeroTwoEight}{Thursday, February 27, 2025}
\newcommand{\commonTitleZeroTwoEight}{MATH 110 Lecture 2.8}
\newcommand{\commonSubtitleZeroTwoEight}{Related Rates}
\newcommand{\commonPSTitleZeroTwoEight}{MATH 110 Problem Set 2.8}

% Lab 7
\newcommand{\commonEventNineteen}{LabSix}
\newcommand{\commonDateLabSix}{Monday, March 3, 2025}
\newcommand{\commonTitleLabSix}{MATH 110 Lab 7}
\newcommand{\commonSubtitleLabSix}{Quiz 5: Sections 2.5, 2.6}

% Section 3.1
\newcommand{\commonEventTwenty}{ZeroThreeOne}
\newcommand{\commonDateZeroThreeOne}{Tuesday, March 4, 2025}
\newcommand{\commonTitleZeroThreeOne}{MATH 110 Lecture 3.1}
\newcommand{\commonSubtitleZeroThreeOne}{Maximum and Minimum Values}
\newcommand{\commonPSTitleZeroThreeOne}{MATH 11 Problem Set 3.1}

% Section 3.2
\newcommand{\commonEventTwentyOne}{ZeroThreeTwo}
\newcommand{\commonDateZeroThreeTwo}{Thursday, March 6, 2025}
\newcommand{\commonTitleZeroThreeTwo}{MATH 110 Lecture 3.2}
\newcommand{\commonSubtitleZeroThreeTwo}{The Mean Value Theorem}
\newcommand{\commonPSTitleZeroThreeTwo}{MATH 110 Problem Set 3.2}

% Lab 8
\newcommand{\commonEventTwentyTwo}{LabSeven}
\newcommand{\commonDateMTTwo}{Monday, March 10, 2025}
\newcommand{\commonDateLabSeven}{Monday, March 10, 2025}
\newcommand{\commonTitleLabSeven}{MATH 110 Lab 8}
\newcommand{\commonSubtitleLabSeven}{Midterm: Chapter 2}

% Section 3.3
\newcommand{\commonEventTwentyThree}{ZeroThreeThree}
\newcommand{\commonDateZeroThreeThree}{Tuesday, March 11, 2025}
\newcommand{\commonTitleZeroThreeThree}{MATH 110 Lecture 3.3}
\newcommand{\commonSubtitleZeroThreeThree}{How Derivatives Affect the
  Shape of a Graph}
\newcommand{\commonPSTitleZeroThreeThree}{MATH 110 Problem Set 3.3}

% Section 3.4
\newcommand{\commonEventTwentyFour}{ZeroThreeFour}
\newcommand{\commonDateZeroThreeFour}{Thursday, March 13, 2025}
\newcommand{\commonTitleZeroThreeFour}{MATH 110 Lecture 3.4}
\newcommand{\commonSubtitleZeroThreeFour}{Limits at Infinity;
  Horizontal Asymptotes}
\newcommand{\commonPSTitleZeroThreeFour}{MATH 110 Problem Set 3.4}

% Lab 9
\newcommand{\commonEventTwentyFive}{LabEight}
\newcommand{\commonDateLabEight}{Monday, March 17, 2025}
\newcommand{\commonTitleLabEight}{MATH 110 Lab 9}
\newcommand{\commonSubtitleLabEight}{Quiz 6: Sections 3.1, 3.2}

% Section 3.5
\newcommand{\commonEventTwentySix}{ZeroThreeFive}
\newcommand{\commonDateZeroThreeFive}{Tuesday, March 18, 2025}
\newcommand{\commonTitleZeroThreeFive}{MATH 110 Lecture 3.5}
\newcommand{\commonSubtitleZeroThreeFive}{Summary of Curve Sketching}
\newcommand{\commonPSTitleZeroThreeFive}{MATH 110 Problem Set 3.5}

% Section 3.7
\newcommand{\commonEventTwentySeven}{ZeroThreeSeven}
\newcommand{\commonDateZeroThreeSeven}{Thursday, March 20, 2025}
\newcommand{\commonTitleZeroThreeSeven}{MATH 110 Lecture 3.7}
\newcommand{\commonSubtitleZeroThreeSeven}{Optimization Problems}
\newcommand{\commonPSTitleZeroThreeSeven}{MATH 110 Problem Set 3.7}

% Lab 10
\newcommand{\commonEventTwentyEight}{LabNine}
\newcommand{\commonDateLabNine}{Monday, March 24, 2025}
\newcommand{\commonTitleLabNine}{MATH 110 Lab 10}
\newcommand{\commonSubtitleLabNine}{Quiz 7: Sections 3.3, 3.4}

% Section 4.1
\newcommand{\commonEventTwentyNine}{ZeroFourOne}
\newcommand{\commonDateZeroFourOne}{Tuesday, March 25, 2025}
\newcommand{\commonTitleZeroFourOne}{MATH 110 Lecture 4.1}
\newcommand{\commonSubtitleZeroFourOne}{Areas and Distances}
\newcommand{\commonPSTitleZeroFourOne}{MATH 110 Problem Set 4.1}

% Section 4.2
\newcommand{\commonEventThirty}{ZeroFourTwo}
\newcommand{\commonDateZeroFourTwo}{Thursday, March 27, 2025}
\newcommand{\commonTitleZeroFourTwo}{MATH 110 Lecture 4.2}
\newcommand{\commonSubtitleZeroFourTwo}{The Definite Integral}
\newcommand{\commonPSTitleZeroFourTwo}{MATH 110 Problem Set 4.2}

% Lab 11
\newcommand{\commonEventThirtyOne}{LabTen}
\newcommand{\commonDateLabTen}{Monday, March 31, 2025}
\newcommand{\commonTitleLabTen}{MATH 110 Lab 11}
\newcommand{\commonSubtitleLabTen}{Quiz 8: Sections 3.5, 3.7}

% Section 4.3
\newcommand{\commonEventThirtyTwo}{ZeroFourThree}
\newcommand{\commonDateZeroFourThree}{Tuesday, April 1, 2025}
\newcommand{\commonTitleZeroFourThree}{MATH 110 Lecture 4.3}
\newcommand{\commonSubtitleZeroFourThree}{The Fundamental Theorem of Calculus}
\newcommand{\commonPSTitleZeroFourThree}{MATH 110 Problem Set 4.3}

% Section 4.4
\newcommand{\commonEventThirtyThree}{ZeroFourFour}
\newcommand{\commonDateZeroFourFour}{Thursday, April 3, 2025}
\newcommand{\commonTitleZeroFourFour}{MATH 110 Lecture 4.4}
\newcommand{\commonSubtitleZeroFourFour}{Indefinite Integrals and the
  Net Change Theorem}
\newcommand{\commonPSTitleZeroFourFour}{MATH 110 Problem Set 4.4}

% Lab 12
\newcommand{\commonEventThirtyFour}{LabEleven}
\newcommand{\commonDateLabEleven}{Monday, April 7, 2025}
\newcommand{\commonTitleLabEleven}{MATH 110 Lab 12}
\newcommand{\commonSubtitleLabEleven}{Quiz 9: Sections 4.1, 4.2}

% Section 4.5
\newcommand{\commonEventThirtyFive}{ZeroFourFive}
\newcommand{\commonDateZeroFourFive}{Tuesday, April 8, 2025}
\newcommand{\commonTitleZeroFourFive}{MATH 110 Lecture 4.5}
\newcommand{\commonSubtitleZeroFourFive}{The Substitution Rule}
\newcommand{\commonPSTitleZeroFourFive}{MATH 110 Problem Set 4.5}

% Section 5.1
\newcommand{\commonEventThirtySix}{ZeroFiveOne}
\newcommand{\commonDateZeroFiveOne}{Thursday, April 10, 2025}
\newcommand{\commonTitleZeroFiveOne}{MATH 110 Lecture 5.1}
\newcommand{\commonSubtitleZeroFiveOne}{Areas Between Curves}
\newcommand{\commonPSTitleZeroFiveOne}{MATH 110 Problem Set 5.1}

% Lab 13
\newcommand{\commonEventThirtySeven}{LabTwelve}
\newcommand{\commonDateLabTwelve}{Monday, April 14, 2025}
\newcommand{\commonTitleLabTwelve}{MATH 110 Review Lab}
\newcommand{\commonSubtitleLabTwelve}{Bonus Quiz 10: Sections 4.3, 4.4}

% Final Class
\newcommand{\commonEventThirtyEight}{FinalClass}
\newcommand{\commonDateFinalClass}{Tuesday, April 15, 2025}
\newcommand{\commonTitleFinalClass}{MATH 110 Review Class}
\newcommand{\commonSubtitleFinalClass}{Answer Questions, Review for Exam}

% Final Exam
\newcommand{\commonEventThirtyNine}{Final}
\newcommand{\commonDateFinal}{Thursday, April 22, 2025}
\newcommand{\commonTitleFinal}{MATH 110 Final Exam}
\newcommand{\commonSubtitleFinal}{Comprehensive Exam: All Sections}

% Orphaned -- no longer part of the course

% Section 2.9
\newcommand{\commonDateZeroTwoNine}{Not part of the course}
\newcommand{\commonTitleZeroTwoNine}{MATH 110 Lecture 2.9}
\newcommand{\commonSubtitleZeroTwoNine}{Linear Approximations and Differentials}
\newcommand{\commonPSTitleZeroTwoNine}{MATH 110 Problem Set 2.9}


% % Introduction
% \newcommand{\commonEventOneDate}{Wednesday, September 8, 2010}
% \newcommand{\commonEventOneDesc}{Introduction to the Course}
% \newcommand{\commonDateZeroZeroZero}{September 8, 2010}
% \newcommand{\commonTitleZeroZeroZero}{MATH 104 Introduction}
% \newcommand{\commonSubtitleZeroZeroZero}{Outline of the Course}

% % Lecture 1
% \newcommand{\commonEventTwoDate}{Friday, September 10, 2010}
% \newcommand{\commonEventTwoDesc}{Lecture 1: Algebra}
% \newcommand{\commonDateZeroZeroOne}{September 10, 2010}
% \newcommand{\commonTitleZeroZeroOne}{MATH 104 Lecture 1}
% \newcommand{\commonSubtitleZeroZeroOne}{Review of Algebra}
% % associated evaluation ... factor this out?
% \newcommand{\commonPSTitleZeroZeroOne}{MATH 104 Problem Set 1}
% \newcommand{\commonEvalZeroZeroOne}{Quiz 1}
% \newcommand{\commonEvalDateZeroZeroOne}{Wednesday, September 15, 2010}

% % Lecture 2
% \newcommand{\commonEventThreeDate}{Monday, September 13, 2010}
% \newcommand{\commonEventThreeDesc}{Lecture 2: Appendix A}
% \newcommand{\commonDateZeroZeroA}{September 13, 2010}
% \newcommand{\commonTitleZeroZeroA}{MATH 104 Lecture 2}
% \newcommand{\commonSubtitleZeroZeroA}{Appendix A: Numbers, Inequalities, 
%   and Absolute Values}
% % associated evaluation ... factor this out?
% \newcommand{\commonPSTitleZeroZeroA}{MATH 104 Problem Set 2}
% \newcommand{\commonEvalZeroZeroA}{Quiz 2}
% \newcommand{\commonEvalDateZeroZeroA}{Wednesday, September 22, 2010}

% % Review 1
% \newcommand{\commonEventFourDate}{Wednesday, September 15, 2010}
% \newcommand{\commonEventFourDesc}{Review 1: Review Algebra; Quiz 1; Review Appendix A}
% \newcommand{\commonDateRZeroOne}{September 15, 2010}
% \newcommand{\commonTitleRZeroOne}{MATH 104 Review 1}
% \newcommand{\commonSubtitleRZeroOne}{Review of Algebra, Appendix A}

% % Lecture 3
% \newcommand{\commonEventFiveDate}{Friday, September 17, 2010}
% \newcommand{\commonEventFiveDesc}{Lecture 3: Appendix B}
% \newcommand{\commonDateZeroZeroB}{September 17, 2010}
% \newcommand{\commonTitleZeroZeroB}{MATH 104 Lecture 3}
% \newcommand{\commonSubtitleZeroZeroB}{Appendix B: Coordinate Geometry and Lines}
% % associated evaluation ... factor this out?
% \newcommand{\commonPSTitleZeroZeroB}{MATH 104 Problem Set 3}
% \newcommand{\commonEvalZeroZeroB}{Quiz 2}
% \newcommand{\commonEvalDateZeroZeroB}{Wednesday, September 22, 2010}

% % Lecture 4
% \newcommand{\commonEventSixDate}{Monday, Sepbember 20, 2010}
% \newcommand{\commonEventSixDesc}{Lecture 4: Appendix C}
% \newcommand{\commonDateZeroZeroC}{September 20, 2010}
% \newcommand{\commonTitleZeroZeroC}{MATH 104 Lecture 4}
% \newcommand{\commonSubtitleZeroZeroC}{Appendix C: Graphs of Second-Degree Equations}
% % associated evaluation ... factor this out?
% \newcommand{\commonPSTitleZeroZeroC}{MATH 104 Problem Set 4}
% \newcommand{\commonEvalZeroZeroC}{Midterm 0}
% \newcommand{\commonEvalDateZeroZeroC}{Wednesday, September 29, 2010}

% % Review 2
% \newcommand{\commonEventSevenDate}{Wednesday, September 22, 2010}
% \newcommand{\commonEventSevenDesc}{Review 2: Review Appendix B; Quiz 2; Review Appendix C}
% \newcommand{\commonDateRZeroTwo}{September 22, 2010}
% \newcommand{\commonTitleRZeroTwo}{MATH 104 Review 2}
% \newcommand{\commonSubtitleRZeroTwo}{Review of Appendices B and C}

% % Lecture 5
% \newcommand{\commonEventEightDate}{Friday, September 24, 2010}
% \newcommand{\commonEventEightDesc}{Lecture 5: Appendix D}
% \newcommand{\commonDateZeroZeroD}{September 24, 2010}
% \newcommand{\commonTitleZeroZeroD}{MATH 104 Lecture 5}
% \newcommand{\commonSubtitleZeroZeroD}{Appendix D: Trigonometry}
% % associated evaluation ... factor this out?
% \newcommand{\commonPSTitleZeroZeroD}{MATH 104 Problem Set 5}
% \newcommand{\commonEvalZeroZeroD}{Midterm 0}
% \newcommand{\commonEvalDateZeroZeroD}{Wednesday, September 29, 2010}

% % Lecture 6
% \newcommand{\commonEventNineDate}{Monday, September 27, 2010}
% \newcommand{\commonEventNineDesc}{Lecture 6: Section 1.1}
% \newcommand{\commonDateZeroOneOne}{September 27, 2010}
% \newcommand{\commonTitleZeroOneOne}{MATH 104 Lecture 6}
% \newcommand{\commonSubtitleZeroOneOne}{Section 1.1: Four Ways to Represent a Function}
% % associated evaluation ... factor this out?
% \newcommand{\commonPSTitleZeroOneOne}{MATH 104 Problem Set 6}
% \newcommand{\commonEvalZeroOneOne}{Quiz 3}
% \newcommand{\commonEvalDateZeroOneOne}{Wednesday, October 6, 2010}

% % Review 3
% \newcommand{\commonEventTenDate}{Wednesday, September 29, 2010}
% \newcommand{\commonEventTenDesc}{Review 3: Review Appendix D; 
%   Self-Assessment Midterm 0}
% \newcommand{\commonDateRZeroThree}{September 29, 2010}
% \newcommand{\commonTitleRZeroThree}{MATH 104 Review 3}
% \newcommand{\commonSubtitleRZeroThree}{Review of Appendix D}

% % Lecture 7
% \newcommand{\commonEventElevenDate}{Friday, October 1, 2010}
% \newcommand{\commonEventElevenDesc}{Lecture 7: Section 1.2}
% \newcommand{\commonDateZeroOneTwo}{October 1, 2010}
% \newcommand{\commonTitleZeroOneTwo}{MATH 104 Lecture 7}
% \newcommand{\commonSubtitleZeroOneTwo}{Section 1.2: Mathematical Models: A Catalog of Essential Functions}
% % associated evaluation ... factor this out?
% \newcommand{\commonPSTitleZeroOneTwo}{MATH 104 Problem Set 7}
% \newcommand{\commonEvalZeroOneTwo}{Quiz 3}
% \newcommand{\commonEvalDateZeroOneTwo}{Wednesday, October 6, 2010}

% % Lecture 8
% \newcommand{\commonEventTwelveDate}{Monday, October 4, 2010}
% \newcommand{\commonEventTwelveDesc}{Lecture 8: Section 1.3}
% \newcommand{\commonDateZeroOneThree}{October 4, 2010}
% \newcommand{\commonTitleZeroOneThree}{MATH 104 Lecture 8}
% \newcommand{\commonSubtitleZeroOneThree}{Section 1.3: New Functions from Old Functions}
% % associated evaluation ... factor this out?
% \newcommand{\commonPSTitleZeroOneThree}{MATH 104 Problem Set 8}
% \newcommand{\commonEvalZeroOneThree}{Quiz 4}
% \newcommand{\commonEvalDateZeroOneThree}{Wednesday, October 13, 2010}

% % Review 4
% \newcommand{\commonEventThirteenDate}{Wednesday, October 6, 2010}
% \newcommand{\commonEventThirteenDesc}{Review 4: Review 1.1, 1.2; Quiz 3}
% \newcommand{\commonDateROneOne}{October 6, 2010}
% \newcommand{\commonTitleROneOne}{MATH 104 Review 4}
% \newcommand{\commonSubtitleROneOne}{Reveiw of 1.1, 1.2}

% % Lecture 9
% \newcommand{\commonEventFourteenDate}{Friday, October 8, 2010}
% \newcommand{\commonEventFourteenDesc}{Lecture 9: Section 1.4}
% \newcommand{\commonDateZeroOneFour}{October 8, 2010}
% \newcommand{\commonTitleZeroOneFour}{MATH 104 Lecture 9}
% \newcommand{\commonSubtitleZeroOneFour}{Section 1.4: Graphing Calculators and Computers}
% % associated evaluation ... factor this out?
% \newcommand{\commonPSTitleZeroOneFour}{MATH 104 Problem Set 9}
% \newcommand{\commonEvalZeroOneFour}{Quiz 4}
% \newcommand{\commonEvalDateZeroOneFour}{Wednesday, October 13, 2010}

% % Thanksgiving holiday
% \newcommand{\commonEventFifteenDate}{Monday, October 11, 2010}
% \newcommand{\commonEventFifteenDesc}{No class: Thanksgiving holiday}

% % Review 5
% \newcommand{\commonEventSixteenDate}{Wednesday, October 13, 2010}
% \newcommand{\commonEventSixteenDesc}{Review 5: Review 1.3, 1.4; Quiz 4}
% \newcommand{\commonDateROneTwo}{October 13, 2010}
% \newcommand{\commonTitleROneTwo}{MATH 104 Review 5}
% \newcommand{\commonSubtitleOneRTwo}{Review of 1.3, 1.4}

% % Lecture 10
% \newcommand{\commonEventSeventeenDate}{Friday, October 15, 2010}
% \newcommand{\commonEventSeventeenDesc}{Lecture 10: Section 1.5}
% \newcommand{\commonDateZeroOneFive}{October 15, 2010}
% \newcommand{\commonTitleZeroOneFive}{MATH 104 Lecture 10}
% \newcommand{\commonSubtitleZeroOneFive}{Section 1.5: Exponential Functions}
% % associated evaluation ... factor this out?
% \newcommand{\commonPSTitleZeroOneFive}{MATH 104 Problem Set 10}
% \newcommand{\commonEvalZeroOneFive}{Quiz 5}
% \newcommand{\commonEvalDateZeroOneFive}{Wednesday, October 20, 2010}

% % Lecture 11
% \newcommand{\commonEventEighteenDate}{Monday, October 18, 2010}
% \newcommand{\commonEventEighteenDesc}{Lecture 11: Section 1.6}
% \newcommand{\commonDateZeroOneSix}{October 18, 2010}
% \newcommand{\commonTitleZeroOneSix}{MATH 104 Lecture 11}
% \newcommand{\commonSubtitleZeroOneSix}{Section 1.6: Inverse Functions and Logarithms}
% % associated evaluation ... factor this out?
% \newcommand{\commonPSTitleZeroOneSix}{MATH 104 Problem Set 11}
% \newcommand{\commonEvalZeroOneSix}{Midterm 1}
% \newcommand{\commonEvalDateZeroOneSix}{Wednesday, October 27, 2010}

% % Review 6
% \newcommand{\commonEventNineteenDate}{Wednesday, October 20, 2010}
% \newcommand{\commonEventNineteenDesc}{Review 6: Review 1.5; Quiz 5; Review 1.6}
% \newcommand{\commonDateROneThree}{October 20, 2010}
% \newcommand{\commonDateZeroOneR}{October 20, 2010}
% \newcommand{\commonTitleROneThree}{MATH 104 Review 6}
% \newcommand{\commonSubtitleROneThree}{Review of 1.5, 1.6}
% % associated evaluation ... factor this out?
% \newcommand{\commonPSTitleZeroOneR}{MATH 104 Problem Set R1}
% \newcommand{\commonEvalZeroOneR}{Midterm 1}
% \newcommand{\commonEvalDateZeroOneR}{Wednesday, October 27, 2010}

% % Lecture 12
% \newcommand{\commonEventTwentyDate}{Friday, October 22, 2010}
% \newcommand{\commonEventTwentyDesc}{Lecture 12: Section 2.1}
% \newcommand{\commonDateZeroTwoOne}{October 22, 2010}
% \newcommand{\commonTitleZeroTwoOne}{MATH 104 Lecture 12}
% \newcommand{\commonSubtitleZeroTwoOne}{Section 2.1: The Tangent and Velocity Problems}
% % associated evaluation ... factor this out?
% \newcommand{\commonPSTitleZeroTwoOne}{MATH 104 Problem Set 12}
% \newcommand{\commonEvalZeroTwoOne}{Quiz 6}
% \newcommand{\commonEvalDateZeroTwoOne}{Wednesday, November 3, 2010}

% % Lecture 13
% \newcommand{\commonEventTwentyOneDate}{Monday, October 25, 2010}
% \newcommand{\commonEventTwentyOneDesc}{Lecture 13: Section 2.2(a)}
% \newcommand{\commonDateZeroTwoTwoa}{October 25, 2010}
% \newcommand{\commonTitleZeroTwoTwoa}{MATH 104 Lecture 13}
% \newcommand{\commonSubtitleZeroTwoTwoa}{Section 2.2(a): The Limit of a Function I}
% % associated evaluation ... factor this out?
% \newcommand{\commonPSTitleZeroTwoTwoa}{MATH 104 Problem Set 13}
% \newcommand{\commonEvalZeroTwoTwoa}{Quiz 6}
% \newcommand{\commonEvalDateZeroTwoTwoa}{Wednesday, November 3, 2010}

% % Midterm Test 1
% % October 27, 2010
% \newcommand{\commonEventTwentyTwoDate}{Wednesday, October 27, 2010}
% \newcommand{\commonEventTwentyTwoDesc}{Midterm Test 1: Chapter 1}

% % Lecture 14
% \newcommand{\commonEventTwentyThreeDate}{Friday, October 29, 2010}
% \newcommand{\commonEventTwentyThreeDesc}{Lecture 14: Section 2.2(b)}
% \newcommand{\commonDateZeroTwoTwob}{October 29, 2010}
% \newcommand{\commonTitleZeroTwoTwob}{MATH 104 Lecture 14}
% \newcommand{\commonSubtitleZeroTwoTwob}{Section 2.2(b): The Limit of a Function II}
% % associated evaluation ... factor this out?
% \newcommand{\commonPSTitleZeroTwoTwob}{MATH 104 Problem Set 14}
% \newcommand{\commonEvalZeroTwoTwob}{Quiz 6}
% \newcommand{\commonEvalDateZeroTwoTwob}{Wednesday, November 3, 2010}

% % Lecture 15
% \newcommand{\commonEventTwentyFourDate}{Monday, November 1, 2010}
% \newcommand{\commonEventTwentyFourDesc}{Lecture 15: Section 2.3}
% \newcommand{\commonDateZeroTwoThree}{November 1, 2010}
% \newcommand{\commonTitleZeroTwoThree}{MATH 104 Lecture 15}
% \newcommand{\commonSubtitleZeroTwoThree}{Section 2.3: Calculating Limits Using the Limit Laws}
% % associated evaluation ... factor this out?
% \newcommand{\commonPSTitleZeroTwoThree}{MATH 104 Problem Set 15}
% \newcommand{\commonEvalZeroTwoThree}{Quiz 7}
% \newcommand{\commonEvalDateZeroTwoThree}{Wednesday, November 10, 2010}

% % Review 7
% \newcommand{\commonEventTwentyFiveDate}{Wednesday, November 3, 2010}
% \newcommand{\commonEventTwentyFiveDesc}{Review 7: Review 2.1, 2.2; Quiz 6; Review 2.3}
% \newcommand{\commonDateRTwoOne}{November 3, 2010}
% \newcommand{\commonTitleRTwoOne}{MATH 104 Review 7}
% \newcommand{\commonSubtitleRTwoOne}{Review of 2.1, 2.2, 2.3}

% % Lecture 16
% \newcommand{\commonEventTwentySixDate}{Friday, November 5, 2010}
% \newcommand{\commonEventTwentySixDesc}{Lecture 16: Section 2.5}
% \newcommand{\commonDateZeroTwoFive}{November 5, 2010}
% \newcommand{\commonTitleZeroTwoFive}{MATH 104 Lecture 16}
% \newcommand{\commonSubtitleZeroTwoFive}{Section 2.5: Continuity}
% % associated evaluation ... factor this out?
% \newcommand{\commonPSTitleZeroTwoFive}{MATH 104 Problem Set 16}
% \newcommand{\commonEvalZeroTwoFive}{Quiz 7}
% \newcommand{\commonEvalDateZeroTwoFive}{Wednesday, November 10, 2010}

% % Lecture 17
% \newcommand{\commonEventTwentySevenDate}{Monday, November 8, 2010}
% \newcommand{\commonEventTwentySevenDesc}{Lecture 17: Section 2.6}
% \newcommand{\commonDateZeroTwoSix}{November 8, 2010}
% \newcommand{\commonTitleZeroTwoSix}{MATH 104 Lecture 17}
% \newcommand{\commonSubtitleZeroTwoSix}{Section 2.6: Limits at Infinity: Horizontal Asymptotes}
% % associated evaluation ... factor this out?
% \newcommand{\commonPSTitleZeroTwoSix}{MATH 104 Problem Set 17}
% \newcommand{\commonEvalZeroTwoSix}{Quiz 8}
% \newcommand{\commonEvalDateZeroTwoSix}{Wednesday, November 17, 2010}

% % Review 8
% \newcommand{\commonEventTwentyEightDate}{Wednesday, November 10, 2010}
% \newcommand{\commonEventTwentyEightDesc}{Review 8: Review 2.5; Quiz 7; Review 2.6}
% \newcommand{\commonDateRTwoTwo}{November 10, 2010}
% \newcommand{\commonTitleRTwoTwo}{MATH 104 Review 8}
% \newcommand{\commonSubtitleRTwoTwo}{Review of 2.5, 2.6}

% % Lecture 18
% \newcommand{\commonEventTwentyNineDate}{Friday, November 12, 2010}
% \newcommand{\commonEventTwentyNineDesc}{Lecture 18: Section 2.7}
% \newcommand{\commonDateZeroTwoSeven}{November 12, 2010}
% \newcommand{\commonTitleZeroTwoSeven}{MATH 104 Lecture 18}
% \newcommand{\commonSubtitleZeroTwoSeven}{Section 2.7: Derivatives and Rates of Change}
% % associated evaluation ... factor this out?
% \newcommand{\commonPSTitleZeroTwoSeven}{MATH 104 Problem Set 18}
% \newcommand{\commonEvalZeroTwoSeven}{Quiz 8}
% \newcommand{\commonEvalDateZeroTwoSeven}{Wednesday, November 17, 2010}

% % Lecture 19
% \newcommand{\commonEventThirtyDate}{Monday, November 15, 2010}
% \newcommand{\commonEventThirtyDesc}{Lecture 19: Section 2.8}
% \newcommand{\commonDateZeroTwoEight}{November 15, 2010}
% \newcommand{\commonTitleZeroTwoEight}{MATH 104 Lecture 19}
% \newcommand{\commonSubtitleZeroTwoEight}{Section 2.8: The Derivative as a Function}
% % associated evaluation ... factor this out?
% \newcommand{\commonPSTitleZeroTwoEight}{MATH 104 Problem Set 19}
% \newcommand{\commonEvalZeroTwoEight}{Midterm 2}
% \newcommand{\commonEvalDateZeroTwoEight}{Wednesday, November 24, 2010}

% % Review 9
% % November 17, 2010
% \newcommand{\commonEventThirtyOneDate}{Wednesday, November 17, 2010}
% \newcommand{\commonEventThirtyOneDesc}{Review 9: Review 2.7; Quiz 8; Review 2.8}
% \newcommand{\commonDateRTwoThree}{November 17, 2010}
% \newcommand{\commonTitleRTwoThree}{MATH 104 Review 9}
% \newcommand{\commonSubtitleRTwoThree}{Review of 2.7, 2.8}

% % Lecture 20
% \newcommand{\commonEventThirtyTwoDate}{Friday, November 19, 2010}
% \newcommand{\commonEventThirtyTwoDesc}{Lecture 20: Section 3.1}
% \newcommand{\commonDateZeroThreeOne}{November 19, 2010}
% \newcommand{\commonTitleZeroThreeOne}{MATH 104 Lecture 20}
% \newcommand{\commonSubtitleZeroThreeOne}{Section 3.1: Derivatives of Polynomials and Exponential Functions}
% % associated evaluation ... factor this out?
% \newcommand{\commonPSTitleZeroThreeOne}{MATH 104 Problem Set 20}
% \newcommand{\commonEvalZeroThreeOne}{Quiz 9}
% \newcommand{\commonEvalDateZeroThreeOne}{Wednesday, December 1, 2010}

% % Lecture 21
% \newcommand{\commonEventThirtyThreeDate}{Monday, November 22, 2010}
% \newcommand{\commonEventThirtyThreeDesc}{Lecture 21: Section 3.2}
% \newcommand{\commonDateZeroThreeTwo}{November 22, 2010}
% \newcommand{\commonTitleZeroThreeTwo}{MATH 104 Lecture 21}
% \newcommand{\commonSubtitleZeroThreeTwo}{Section 3.2: The Product and Quotient Rules}
% % associated evaluation ... factor this out?
% \newcommand{\commonPSTitleZeroThreeTwo}{MATH 104 Problem Set 21}
% \newcommand{\commonEvalZeroThreeTwo}{Quiz 9}
% \newcommand{\commonEvalDateZeroThreeTwo}{Wednesday, December 1, 2010}

% % Midterm Test 2
% \newcommand{\commonEventThirtyFourDate}{Wednesday, November 24, 2010}
% \newcommand{\commonEventThirtyFourDesc}{Midterm Test 2: Chapter 2}

% % Lecture 22
% \newcommand{\commonEventThirtyFiveDate}{Friday, November 26, 2010}
% \newcommand{\commonEventThirtyFiveDesc}{Lecture 22: Section 3.3}
% \newcommand{\commonDateZeroThreeThree}{November 26, 2010}
% \newcommand{\commonTitleZeroThreeThree}{MATH 104 Lecture 22}
% \newcommand{\commonSubtitleZeroThreeThree}{Section 3.3: Derivatives of Trigonometric Functions}
% % associated evaluation ... factor this out?
% \newcommand{\commonPSTitleZeroThreeThree}{MATH 104 Problem Set 22}
% \newcommand{\commonEvalZeroThreeThree}{Quiz 9}
% \newcommand{\commonEvalDateZeroThreeThree}{Wednesday, December 1, 2010}

% % Lecture 23
% \newcommand{\commonEventThirtySixDate}{Monday, November 29, 2010}
% \newcommand{\commonEventThirtySixDesc}{Lecture 23: Section 3.4}
% \newcommand{\commonDateZeroThreeFour}{November 29, 2010}
% \newcommand{\commonTitleZeroThreeFour}{MATH 104 Lecture 23}
% \newcommand{\commonSubtitleZeroThreeFour}{Section 3.4: The Chain Rule}
% % associated evaluation ... factor this out?
% \newcommand{\commonPSTitleZeroThreeFour}{MATH 104 Problem Set 23}
% \newcommand{\commonEvalZeroThreeFour}{the final exam}
% \newcommand{\commonEvalDateZeroThreeFour}{Monday, December 13, 2010}

% % Review 10
% \newcommand{\commonEventThirtySevenDate}{Wednesday, December 1, 2010}
% \newcommand{\commonEventThirtySevenDesc}{Review 10: Review 3.1, 3.2, 3.3; Quiz 9}
% \newcommand{\commonDateRThreeTwo}{December 1, 2010}
% \newcommand{\commonTitleRThreeTwo}{MATH 104 Review 10}
% \newcommand{\commonSubtitleRThreeTwo}{Review of 3.1, 3.2, 3.3}

% % Lecture 24
% \newcommand{\commonEventThirtyEightDate}{Friday, December 3, 2010}
% \newcommand{\commonEventThirtyEightDesc}{Lecture 24: Section 3.5}
% \newcommand{\commonDateZeroThreeFive}{December 3, 2010}
% \newcommand{\commonTitleZeroThreeFive}{MATH 104 Lecture 24}
% \newcommand{\commonSubtitleZeroThreeFive}{Section 3.5: Implicit Differentiation}
% % associated evaluation ... factor this out?
% \newcommand{\commonPSTitleZeroThreeFive}{MATH 104 Problem Set 24}
% \newcommand{\commonEvalZeroThreeFive}{the final exam}
% \newcommand{\commonEvalDateZeroThreeFive}{Monday, December 13, 2010}

% % Lecture 25
% \newcommand{\commonEventThirtyNineDate}{Monday, December 6, 2010}
% \newcommand{\commonEventThirtyNineDesc}{Lecture 25: Section 3.6}
% \newcommand{\commonDateZeroThreeSix}{December 6, 2010}
% \newcommand{\commonTitleZeroThreeSix}{MATH 104 Lecture 25}
% \newcommand{\commonSubtitleZeroThreeSix}{Section 3.6: Derivatives of Logarithmic Functions}
% % associated evaluation ... factor this out?
% \newcommand{\commonPSTitleZeroThreeSix}{MATH 104 Problem Set 25}
% \newcommand{\commonEvalZeroThreeSix}{the final exam}
% \newcommand{\commonEvalDateZeroThreeSix}{Monday, December 13, 2010}

% % Review 11
% \newcommand{\commonEventFortyDate}{Wednesday, December 8, 2010}
% \newcommand{\commonEventFortyDesc}{(Bonus) Review 11: Review 3.4, 3.5, 3.6}
% \newcommand{\commonDateRThreeThree}{December 8, 2010}
% \newcommand{\commonTitleRThreeThree}{MATH 104 (Bonus) Review 11}
% \newcommand{\commonSubtitleRThreeThree}{Review of 3.4, 3.5, 3.6}

% % Final Exam
% % December 13, 2010
% \newcommand{\commonEventFinalDate}{Monday, December 13, 2010}
% \newcommand{\commonEventFinalDesc}{MATH 104 Final Exam}

%%% Local variables:
%%% mode: latex
%%% TeX-master: "MATH110-Syllabus.tex"
%%% End:

\usepackage{mathptmx}
\usepackage{multirow}
\usepackage{tikz}

\newcommand{\ds}{\displaystyle}

\mode<article>{}
\mode<presentation>{\usetheme{\commonTheme}\usecolortheme{\commonColorTheme}}

\title{\commonTitleZeroZeroOne}
\subtitle{\commonSubtitleZeroZeroOne}
\author{\commonAuthor}
\institute{\commonInstitute}
\date{\commonDateZeroZeroOne}

\begin{document}

%\section*{Outline}

\begin{frame}
  \titlepage
\end{frame}

\begin{frame}
  \frametitle{Contents}
  \tableofcontents
\end{frame}


\section{Review of Algebra}

\subsection{Arithmetic Operations}

% EJD: nice illustration of commutativity of addition lengths 2+3
% EJD: illustrate how commuting is "moving"
\begin{frame}
  \frametitle{Commutativity of Addition}
  \begin{itemize}[<+->]
  \item If I add two numbers, say 2 and 3, the answer does not depend
    on the order in which I add the numbers.
  \item $2+3=5$, but also $3+2=5$.
  \item In other words we can write $2+3=3+2$.
  \item The above result doesn't depend on the numbers 2 and 3.  For example,
    I also have $4+7=7+4$, and so on.
  \item We say that the operation of addition is \textit{commutative}.
  \end{itemize}
\end{frame}

% EJD: illustrate by sliding lengths a+b
\begin{frame}
  \frametitle{Commutative Law for Addition}
  \begin{itemize}[<+->]
  \item We can express that more general result using \textit{algebraic}
    notation: $a+b=b+a$ for any numbers $a$ and $b$.
  \item For example choosing $a=2$ and $b=3$ gives us the result $2+3=3+2$,
    and choosing $a=4$ and $b=7$ gives us the result $4+7=7+4$.
  \item The notation $a+b=b+a$ is very powerful; it stands for an infinite
    number of different particular results.
  \item We call that expression the \textit{commutative law for addition}.
  \end{itemize}
\end{frame}

% EJD: illustrate by rotating rectangle 3x5
\begin{frame}
  \frametitle{Commutativity of Multiplication}
  \begin{itemize}[<+->]
  \item If I multiply two numbers, say 3 and 5, the answer does not depend
    on the order in which I multiply the numbers.
  \item We have $3\times 5=15$, but we also have $5\times 3=15$
  \item In other words, we can write $5\times 3 = 3\times 5$.
  \item The above result does not depend on the particular numbers $3$
    and $5$.  For example, we have $2\times 6=6\times 2$, and so on,
    for any two numbers I choose.
  \item We say that the operation of multiplication is also commutative.
  \end{itemize}
\end{frame}

% EJD: illustrate by rotating a rectangle sides axb
\begin{frame}
  \frametitle{Commutative Law for Multiplication}
  \begin{itemize}[<+->]
  \item We can express that more general result using \textit{algebraic}
    notation: $a\times b=b\times a$ for any numbers $a$ and $b$.
  \item As a short form, we usually drop the $\times$ if we can do so without
    confusion, so we usually write $ab=ba$.
  \item For example choosing $a=3$ and $b=5$ gives us the result $3\times 5
    =5\times 3$,
    and choosing $a=2$ and $b=6$ gives us the result $2\times 6=6\times 2$.
  \item The notation $ab=ba$ is very powerful; it stands for an infinite
    number of different particular results.
  \item We call that expression the \textit{commutative law for
    multiplication}.
  \end{itemize}
\end{frame}

% EJD put law in coloured box to emphasize
\begin{frame}
  \frametitle{The Commutative Laws}
  \begin{itemize}[<+->]
  \item In summary, we have the commutative law of addition:
    \begin{equation*}
      a+b=b+a
    \end{equation*}
    for any numbers $a$ and $b$
  \item And the commutative law of multiplication
    \begin{equation*}
      ab=ba
    \end{equation*}
  \item We obviously don't have to use the letters $a$ and $b$.  So we
    could write the \textit{commutative laws} as
    \begin{equation*}
      x+y=y+x
    \end{equation*}
    and
    \begin{equation*}
      xy=yx
    \end{equation*}
  \item That is our first result in algebra.
  \end{itemize}
\end{frame}

% EJD: diagram of adding three lengths in a row
% EJD: diagram of rotating box
\begin{frame}
  \frametitle{Associative Laws}
  \begin{itemize}[<+->]
  \item Similarly, the order in which we add three numbers does not
    matter.
  \item For example, $(3+6)+5=3+(6+5)$.
  \item In general, we write $(x+y)+z=x+(y+z)$.
  \item Furthermore, the order in which we multiply three numbers does
    not matter.
  \item For example, $(2\times 5)\times 3 = 2\times (5\times 3)$.
  \item In general, we write $(x\times y)\times z= x\times(y\times
    z)$, or in short form $(xy)z = x(yz)$.
  \end{itemize}
\end{frame}

% EJD: diagram of putting rectangles with sides axb and axc together
\begin{frame}
  \frametitle{Distributive Laws}
  \begin{itemize}[<+->]
  \item The previous laws applied to just one operation, addition or
    multiplication. 
  \item We also have a law which involves both operations
    simultaneously, the distributive law.
  \item Adding first and then multiplying is the same as multiplying
    first and then adding.
  \item More precisely, in symbols, we write \(x(y+z) = (xy) + (xz)\).
  \item Order of operations (BEDMAS) says that multiplication is done
    before addition so we can drop the brackets on the right hand side
    (RHS).
  \item The \textit{distributive law} is usually written
    \(x(y+z) = xy+xz\).
  \end{itemize}
\end{frame}

% EJD: picture to show how expanding is the distributive law three times
\begin{frame}
  \frametitle{Expanding}
  \begin{itemize}[<+->]
  \item The distributive law is the key to an important process in
    algebra known as \textit{expanding}.
  \item Consider the expression $3(t+2)$.
  \item Expanding means applying the distributive law to remove the
    brackets: $3(t+2) = 3\times t + 3\times 2 = 3t+6$.
  \item Now consider $(s+3)(t+2)$.
  \item Apply the distributive law: $(s+3)(t+2)=(s+3)t+(s+3)2$.
  \item Apply it again: $(s+3)t + (s+3)2 = st+3t+(s+3)2$.
  \item And again: $st+3t+(s+3)2=st+3t+2s+6$.
  \end{itemize}
\end{frame}

\begin{frame}
  \frametitle{Simplifying}
  \begin{itemize}[<+->]
  \item We can apply any of the laws we know to simplify expressions.
  \item Consider $4-3x+6$.
  \item We can write $4-3x+6 = 4 + (-3\times x) + 6$.
  \item Applying the commutative law of addition, we have
    $4+(-3\times x) + 6 = 4 + 6 + (-3\times x)$.
  \item We can add the $4$ and $6$ to get $10$: $4+6+(-3\times
    x)=10+(-3\times x)$.
  \item By Order of Operations we can drop the brackets: $10+(-3\times
    x) = 10-3x$.
  \item In summary, $4-3x+6=10-3x$.
  \end{itemize}
\end{frame}

% Nice picture for multiplying fractions
\begin{frame}
  \frametitle{Multiplying Fractions}
  \begin{itemize}[<+->]
  \item We also need to remember rules for dealing with fractions.
  \item The simplest operation we can perform on fractions is
    multiplying them.
  \item The rule for multiplying fractions is
    \begin{equation*}
      \frac{a}{b} \times \frac{c}{d} = \frac{ac}{bd}
    \end{equation*}
  \item We just \textit{multiply the numerators and multiply the
    denominators} to get the answer.
  \end{itemize}
\end{frame}

% EJD: diagram illustrating division of fractions
\begin{frame}
  \frametitle{Dividing Fractions}
  \begin{itemize}[<+->]
  \item The next simplest operation for fractions is division.
  \item To divide fractions we \textit{invert and multiply}:
    \begin{equation*}
      \frac{a}{b} \div \frac{c}{d} = \frac{a}{b} \times \frac{d}{c}
      = \frac{ad}{bc}
    \end{equation*}
  \item Sometimes we see the same procedure in slightly different
    notation:
    \begin{equation*}
      \frac{a/b}{c/d} = \frac{a}{b} \times \frac{d}{c} = \frac{ad}{bc}
    \end{equation*}
  \end{itemize}
\end{frame}

\begin{frame}
  \frametitle{Negating Fractions}
  \begin{itemize}[<+->]
  \item Negating a fraction is the same thing as multiplying it by
    \begin{equation*}
      -1 = \frac{-1}{1} = \frac{1}{-1}
    \end{equation*}
  \item So all three of the following are negatives of $2/3$:
    \begin{equation*}
      -\frac{2}{3} = \frac{-2}{3} = \frac{2}{-3}
    \end{equation*}
  \end{itemize}
\end{frame}

\begin{frame}
  \frametitle{Equivalent Fractions}
  \begin{itemize}[<+->]
  \item There are many different ways of representing the number 1 as
    a fraction:
    \begin{equation*}
      1 = \frac{1}{1} = \frac{2}{2} = \frac{7}{7} = \frac{-3}{-3} = \cdots
    \end{equation*}
  \item Multiplying any fraction by any of those representations of 1
    gives an equivalent fraction:
    \begin{equation*}
      \frac{2}{3} = \frac{4}{6} = \frac{14}{21} = \frac{-6}{-9} =
      \cdots
    \end{equation*}
  \end{itemize}
\end{frame}

\begin{frame}
  \frametitle{Adding Fractions I}
  \begin{itemize}[<+->]
  \item The rule for adding fractions is surprisingly more complicated
    than the rule for multiplying fractions.
  \item To add fractions, we must find a common denominator.
  \item We do that by finding appropriate equivalent fractions.
  \item For example, to add $2/3$ and $4/7$, we find fractions
    equivalent to $2/3$, and fractions equivalent to $4/7$, looking
    for a common denominator.
    \begin{equation*}
      \frac{2}{3} = \frac{4}{6} = \cdots = \frac{14}{21} = \cdots
    \end{equation*}
    \begin{equation*}
      \frac{4}{7} = \frac{8}{14} = \frac{12}{21} = \cdots
    \end{equation*}
  \item We found a common denominator, $21$.
  \end{itemize}
\end{frame}

\begin{frame}
  \frametitle{Adding Fractions II}
  \begin{itemize}[<+->]
  \item Now we can add the two fractions:
    \begin{equation*}
      \frac{2}{3} + \frac{4}{7}
      = \frac{14}{21} + \frac{12}{21} 
      = \frac{26}{21}
    \end{equation*}
  \item Note that, with addition, we just add the numerators and we
    leave the denominators alone.
  \item Note that we can always find a common denominator:
    \begin{equation*}
      \frac{a}{b} + \frac{c}{d} 
      = \frac{ad}{bd} + \frac{bc}{bd}
      = \frac{ad+bc}{bd}
    \end{equation*}
  \item That is a general formula for adding fractions.
  \end{itemize}
\end{frame}

\begin{frame}
  \frametitle{Subtracting Fractions}
  \begin{itemize}[<+->]
  \item We follow a similar procedure to subtract fractions.
  \item We find a common denominator and then subtract only the
    numerators: 
    \begin{equation*}
      \frac{a}{b} - \frac{c}{d} = \frac{ad}{bd} - \frac{bc}{bd} =
      \frac{ad-bc}{bd} 
    \end{equation*}
  \item It's better to remember the procedure for adding and
    subtracting fractions, rather than the formula.
  \end{itemize}
\end{frame}

\subsection{Factoring and Quadratics}

% EJD: show like a chemical equation, maybe
\begin{frame}
  \frametitle{Expanding and Factoring}
  \begin{itemize}[<+->]
  \item The opposite of \textit{expanding} is \textit{factoring}.
  \item If I start with an expression like $2x(x+3)$ I can expand it
    to get \only<3->{$2x^2+6x$}
  \item<4-> On the other hand, if I start with $2x^2+6x$ I can factor
    it to get $2x(x+3)$.
  \item Sometimes expanding is appropriate, and sometimes factoring is
    appropriate.
  \item It's hard to give a rule to tell you which is better in a
    given situation.  You need to use your judgment.
  \end{itemize}
\end{frame}

\begin{frame}
  \frametitle{Factoring Quadratics}
  \begin{itemize}[<+->]
  \item Many of the examples used in our course will require factoring
    quadratics.  
  \item A quadratic is a polynomial with highest degree term of degree
    2, for example $x^2+5x+6$ or $x^2-9$ or $-3x^2+6x-3$.
  \item Often you can guess how to factor a quadratic of the form
    $x^2+ax+b$.
  \item You need to find two numbers that add to give $a$ and multiply
    to give $b$.
  \item For the example $x^2+5x+6$, you would eventually guess $2$ and
    $3$: their sum is $5$ and their product is $6$.
  \item So we guess $x^2+5x+6=(x+2)(x+3)$.  
  \item Check by expanding the RHS.
  \end{itemize}
\end{frame}

% EJD: picture of difference of squares
\begin{frame}
  \frametitle{Difference of Squares}
  \begin{itemize}[<+->]
  \item One common case of quadratic polynomial is a difference of
    squares.
  \item For example, $x^2$ is a square and $9=3^2$ is a square, and
    their difference is $x^2-9$.
  \item A difference of squares can always be factored by taking the
    sum and difference of the square roots of the squares.
  \item For example, $x^2-9 = (x+3)(x-3)$.
  \item Again, check by expanding the RHS.
  \end{itemize}
\end{frame}

% EJD: animate cancellation
\begin{frame}
  \frametitle{Difference of Cubes}
  \begin{itemize}[<+->]
  \item There is also a similar pattern for factoring a difference of
    cubes.
  \item A difference of cubes is a polynomial of the form $x^3-a^3$,
    for example, $x^3-8=x^3-2^3$.
  \item The pattern is $x^3-a^3=(x-a)(x^2+ax+a^2)$.
  \item We can check by expanding the RHS: 
    \begin{equation*}
      (x-a)(x^2+ax+a^2) = x^3 + ax^2 + a^2x -ax^2 -a^2x - a^3 = x^3 - a^3
    \end{equation*}
  \item For example, $x^3-8=(x-2)(x^2+2x+4)$.  (Check!)
  \end{itemize}
\end{frame}

% EJD: animate step by step
\begin{frame}
  \frametitle{Sum of Cubes}
  \begin{itemize}[<+->]
  \item We can't factor a sum of squares like $x^2+9$
  \item However, we can factor a sum of cubes using a trick.
  \item The trick is based on this observation: $b^3=-(-b^3)=-(-b)^3$
  \item So a sum of cubes can be written as a difference of cubes:
    \begin{equation*}
      x^3+b^3 = x^3-(-b)^3 = x^3-a^3
    \end{equation*}
    where $a = -b$.
  \item Using our difference of cubes formula, we have
    $x^3-a^3=(x-a)(x^2+ax+a^2)$
  \item Putting back $-b$ for $a$ we have
    $x^3+b^3=(x+b)(x^2-bx+b^2)$.  Check!
  \end{itemize}
\end{frame}

% EJD: emphasize minus sign
\begin{frame}
  \frametitle{The Factor Theorem}
  \begin{itemize}[<+->]
  \item There is a connection between the roots of a polynomial and
    its factors.
  \item Note that $2$ is a root of the polynomial $x^2-5x+6$, i.e.,
    when we substitute $2$ for $x$, the result is $0$.
  \item $2^2-5(2)+6=4-10+6=0$
  \item That means that $x-2$ should be a factor of $x^2-5x+6$.
  \item Indeed it is: check $x^2-5x+6=(x-2)(x-3)$
  \end{itemize}
\end{frame}

% EJD: polynomial - base b analogy; long division
\begin{frame}
  \frametitle{Polynomial Division I}
  \begin{itemize}[<+->]
  \item If we know one factor of a number, we can find the other
    factor by division.
  \item For example, we know that 2 is a factor of 26, we can find the
    other factor by dividing: $26/2=13$.
  \item We write $26=2\times 13$
  \item We can do the same with polynomials.
  \end{itemize}
\end{frame}

\begin{frame}
  \frametitle{Polynomial Division II}
  \begin{itemize}[<+->]
  \item I recommend you divide polynomials as follows.  First Write
    down the form of the factorization: $x^2-5x+6=(x-2)(x+a)$
  \item Expand and gather the RHS:
    $(x-2)(x+a)=x^2-2x+ax-2a=x^2+(a-2)x+(-2a)$
  \item Compare: $x^2-5x+6=x^2+(a-2)x+(-2a)$ means $a-2=-5$ means
    $a=-3$, so the other factor should be $x+(-3)=x-3$
  \item Check that $x^2-5x+6=(x-2)(x-3)$
  \end{itemize}
\end{frame}

% EJD: better argument: x^2+6x = x^2+6x+9-9
% EJD: graphical illustration
\begin{frame}
  \frametitle{Completing the Square I}
  \begin{itemize}[<+->]
  \item An important technique for dealing with quadratic polynomials
    is \textit{completing the square}.
  \item Consider the polynomial $x^2+6x+5$.
  \item We want to find a square that looks as much like that
    polynomial as possible.
  \item We square $(x+a)$ to obtain $(x+a)^2=x^2+2ax+a^2$.
  \item The first term is correct.  For the second term to be correct
    we need $2a=6$, $a=6/2=3$.
  \end{itemize}
\end{frame}

\begin{frame}
  \frametitle{Completing the Square II}
  \begin{itemize}[<+->]
  \item So we have $x^2+6x+5=(x+3)^2+\mbox{other junk}$
  \item We figure out what the other junk is by expanding and
    simplifying: $x^2+6x+5=x^2+6x+9+\mbox{other junk}$ so
    $-4=\mbox{other junk}$
  \item Altogether we have $x^2+6x+5=(x+3)^2-4$
  \item Now that the quadratic is in that form, we can do other useful
    things with it, e.g., factoring it as a difference of squares.
  \item See the supplement for the procedure to follow when the
    coefficient of $x^2$ is not $1$, e.g., for $2x^2+8x+6$.
  \end{itemize}
\end{frame}

% EJD: highlight quadratic formula
\begin{frame}
  \frametitle{The Quadratic Formula}
  \begin{itemize}[<+->]
  \item Complete the square for the general quadratic $ax^2+bx+c$
  \item We obtain 
    \begin{equation*}
      ax^2+bx+c 
      = a\left(x +\frac{b}{2a}\right)^2 -
      \left(\frac{b^2}{4a}-c\right)
    \end{equation*}
  \item Now if we factor by diffence of squares we get the roots of a
    general quadratic:
    \begin{equation*}
      x = \frac{-b\pm \sqrt{b^2-4ac}}{2a}
    \end{equation*}
  \item That is the quadratic formula, and it can save you a lot of
    work if you memorize it.
  \end{itemize}
\end{frame}

% EJD: diagrams of parabolas ... also show relationship between graph
% and roots above
\begin{frame}
  \frametitle{The Discriminant}
  \begin{itemize}[<+->]
  \item Note that in the quadratic formula, we are asked to take the
    square root of the expresson $b^2-4ac$.
  \item If that expression is negative, we can't take the square root,
    so the quadratic polynomial $ax^2+bx+c$ has no roots in that case.
  \item If that expression is zero, the quadratic equation has two
    identical roots.
  \item If that expression is positive, the quadratic equation has to
    different roots.
  \item Because the expression $b^2-4ac$ helps us discriminate between
    those three cases, it is called the discriminant of the quadratic
    equation. 
  \end{itemize}
\end{frame}

\subsection{Exponents and the Binomial Theorem}

% EJD: Better layout
\begin{frame}
  \frametitle{Powers of a Variable}
  \begin{itemize}[<+->]
  \item We have already seen a short form for $xx$, namely $x^2$.
    Similarly a short form for $xxx$ is $x^3$, and so on.
  \item Following that pattern, $x^1$ should be $x$.
  \item But what is $x^0$?  Let's run the pattern in reverse to find
    out.
  \item To go from $x^3$ to $x^2$, we divide by $x$.
  \item To go from $x^2$ to $x^1$, we divide by $x$.
  \item To go from $x^1$ to $x^0$, we should divide by $x$.  
  \item $x/x=1$ so $x^0$ is $1$.
  \item Similarly, $x^{-1}$ is $1/x$, $x^{-2}$ is $(1/x)/x = 1/x^2$,
    and so on.
  \end{itemize}
\end{frame}

\begin{frame}
  \frametitle{The Binomial Theorem}
  \begin{itemize}[<+->]
  \item Sometimes we have do deal with powers of more complicated
    expressions.
  \item For example, $(x+y)^2=x^2+2xy+y^2$.
  \item Multiplying both sides again by $x+y$, and expanding and
    gathering the RHS,
    \begin{equation*}
      (x+y)^3 = (x^2+2xy+y^2)(x+y) = \cdots = x^3+3x^2y+3xy^2+y^3
    \end{equation*}
  \item Doing that again give
    \begin{equation*}
      (x+y)^4 = x^4 + 4x^3y + 6x^2y^2 + 4xy^3 + y^4
    \end{equation*}
  \item There is a pattern to these results called the
    \textit{binomial theorem}.  See the textbook supplement for details.
  \end{itemize}
\end{frame}

\begin{frame}
  \frametitle{Laws of Exponents}
  \begin{itemize}[<+->]
  \item Consider $x^2 \times x^3$.  
  \item That is the product of two $x$'s times the product of three
    $x$'s.
  \item The result should be $x^5$.  But note $2+3=5$.
  \item In general, we have $x^m \times x^n=x^{m+n}$.
  \item Similarly $x^m/x^n=x^{m-n}$.
  \item Also $(x^m)^n$ is $n$ collections of $m$ $x$'s, so we have
    $(x^m)^n = x^{mn}$.
  \item Another common situation is like $(xy)^2=xyxy=xxyy=x^2y^2$.
  \item In general, $(xy)^m = x^my^m$.
  \item If you think about it, you'll see that those results hold for
    negative values of $m$ and/or $n$ as well.
  \end{itemize}
\end{frame}

\begin{frame}
  \frametitle{Fractional Powers}
  \begin{itemize}[<+->]
  \item We know what $x^n$ means for $n=0,1,2,\ldots$ and
    $n=-1,-2,-3,\ldots$.
  \item Now we would like to figure out what $x^n$ means for
    fractional values of $n$, for example $n=1/2$.
  \item We do that by crossing our fingers and hoping that the laws of
    exponents will apply to fractional exponents.
  \item If that's the case then $(x^{1/2})^2$ should be
    $x^{(1/2)\times 2} = x^1$.
  \item In other words, $x^{1/2}$ should be $\sqrt{x}$.
  \item It turns out that everything works out fine if we say
    $x^{1/2}=\sqrt{x}$, $x^{1/3}=\sqrt[3]{x}$, and so on.
  \end{itemize}
\end{frame}

% EJD: exhortation in red
\begin{frame}
  \frametitle{Radicals}
  \begin{itemize}[<+->]
  \item It is better to write $x^{1/2}$ instead of $\sqrt{x}$, but the
    latter expression is very common so we should learn how to deal
    with it.
  \item For example, by the laws of exponents, we have
    \begin{equation*}
      \sqrt{xy} = (xy)^{1/2} = x^{1/2} y^{1/2} = \sqrt{x} \sqrt{y}
    \end{equation*}
  \item NOTE: you \textit{cannot} write
    \begin{equation*}
      \sqrt{x+y} = \sqrt{x} + \sqrt{y}
    \end{equation*}
  \item That is wrong, wrong, wrong!  Try $x=9$ and $y=16$.
  \item (You could handle $\sqrt{x+y}$ using an extended version of the
    binomial theorem, but that is beyond our powers at the moment.)
  \end{itemize}
\end{frame}

\begin{frame}
  \frametitle{Rationalizing}
  \begin{itemize}[<+->]
  \item There is a useful trick with radicals we should know.  Note
    that the difference of squares factorization gives
    \begin{equation*}
      (\sqrt{x}+\sqrt{y})(\sqrt{x}-\sqrt{y}) = x-y
    \end{equation*}
  \item Suppose we want to simplify the numerator in an expression
    like
    \begin{equation*}
      \frac{\sqrt{x}+\sqrt{y}}{4}
    \end{equation*}
  \item We can do that (at the expense of making the denominator more
    complicated) by multiplying by a specially chosen representation
    of $1$ based on the \textit{conjugate radical} $\sqrt{x}-\sqrt{y}$.
  \item We have
    \begin{equation*}
      \frac{\sqrt{x}+\sqrt{y}}{4} 
      = \frac{\sqrt{x}+\sqrt{y}}{4} \times
      \frac{\sqrt{x}-\sqrt{y}}{\sqrt{x}-\sqrt{y}} 
      = \frac{x-y}{4(\sqrt{x}-\sqrt{y})}
    \end{equation*}
  \end{itemize}
\end{frame}

%% \section{Examples and Exercises}

%% \begin{frame}
%%   \frametitle{Examples}
%% \end{frame}

%% \begin{frame}
%%   \frametitle{Exercises}
%%   The following exercises from the Review of Algebra handout will help you
%%   understand the material.
%%   \begin{itemize}
%%   \item C-level: 
%%   \item B-level: 
%%   \item A-level: 
%%   \end{itemize}
%% \end{frame}


\end{document}

