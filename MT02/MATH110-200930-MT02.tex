\documentclass[12pt]{article}
%\documentclass[12pt,legalpaper]{article}
\usepackage{fullpage}
\usepackage{amsmath,amssymb}
\usepackage{mathptmx}
%\usepackage{fancyheadings}
\usepackage{fancyhdr}
\usepackage{lastpage}
\usepackage[aux]{rerunfilecheck}

%\setlength{\textheight}{11.5in}

\reversemarginpar

\newcommand{\myimp}{\Rightarrow}
\newcommand{\myiff}{\Leftrightarrow}
\newcommand{\mynot}{\neg}
\newcommand{\myor}{\vee}
\newcommand{\myand}{\wedge}
\newcommand{\ds}{\displaystyle}

\DeclareSymbolFont{AMSb}{U}{msb}{m}{n}
\DeclareMathSymbol{\N}{\mathbin}{AMSb}{"4E}
\DeclareMathSymbol{\Z}{\mathbin}{AMSb}{"5A}
\DeclareMathSymbol{\R}{\mathbin}{AMSb}{"52}
\DeclareMathSymbol{\Q}{\mathbin}{AMSb}{"51}
\DeclareMathSymbol{\I}{\mathbin}{AMSb}{"49}
\DeclareMathSymbol{\C}{\mathbin}{AMSb}{"43}

\pagestyle{fancy}
\lhead{MATH110--S01--S02 200930                    \\ 
       Midterm Test 2 \hspace{1.25in} Page\ \thepage\ of \pageref{LastPage} \\ 
       Time: 50 minutes                             \\
       \quad}
%\chead{\quad \\ Page\ \thepage\ of \pageref{LastPage} \\ \quad}
\chead{}
\rhead{Name: \underline{\hspace{2in}}        \\ 
       Student \#: \underline{\hspace{2in}}  \\ 
       Section: \underline{\hspace{2in}}     \\
       \quad}
\cfoot{}
\addtolength{\headheight}{\baselineskip}
\addtolength{\headheight}{\baselineskip}
\addtolength{\headheight}{\baselineskip}
\addtolength{\headheight}{\baselineskip}
\renewcommand{\headrulewidth}{0pt}
\fancypagestyle{plain}{%
  \lhead{}
  \chead{FIRST NATIONS UNIVERSITY OF CANADA           \\
    DEPARTMENT OF SCIENCE                    \\
    MATH110--S01--S02 200930
  }
  \rhead{}
  \cfoot{Page\ \thepage\ of \pageref{LastPage}}
}

\title{Midterm Test 2}
\author{Edward Doolittle}

\begin{document}
\thispagestyle{plain}
%\maketitle

\begin{center}
  \quad\\
  \LARGE{Midterm Test 2}
\end{center}

\begin{flushleft}
\quad\\
Time: 50 minutes                \hfill       Name: \underline{\hspace{2in}}  \\
Instructors:                    \hfill Student \#: \underline{\hspace{2in}}  \\
\quad Dr. Edward Doolittle      \hfill    Section: \underline{\hspace{2in}}  \\
\end{flushleft}


\noindent
You have 50 minutes to do each of the following questions.
The test is worth a total of 50 marks.
Please\marginpar{\small(marks)} justify your conclusions and
show all your work.
A non-programmable calculator of approved type is permitted.  No other aids
are permitted.
Use the backs of the pages for rough work.

\begin{enumerate}
%%%%%%%%%%%%%%%%%%%%%%%%%%%%%%%%%%%%%%%%%%%%%%%%%%%%%%%%%%%%%% begin page 1 %%%
\item Find\marginpar{\small(6)}
  the derivative of the function $f(x)=3x^2-x+1$ from first principles.
\vfill
\newpage
%%%%%%%%%%%%%%%%%%%%%%%%%%%%%%%%%%%%%%%%%%%%%%%%%%%%%%%%%%%%%% begin page 2 %%%
\item Find the following derivatives.  Do not simpify!
  \begin{enumerate}
  \item $\ds y'$\marginpar{\small(5)}
    where 
    $\ds y = \tan^3 (2x)$
\vfill
  \item $\ds\frac{dy}{dx}$\marginpar{\small (6)}
    where
    $\ds y\cos x = x\sin y$
\vfill
  \item The \emph{second derivative} $\ds g''$\marginpar{\small(6)}
    where 
    $\ds g(t)=\frac{t+1}{t-1}$
\vfill
  \end{enumerate}
\newpage
%%%%%%%%%%%%%%%%%%%%%%%%%%%%%%%%%%%%%%%%%%%%%%%%%%%%%%%%%%%%%% begin page 3 %%%
\item Find\marginpar{\small(6)}
  the differential $dy$ of $\ds y=\sqrt{x^2-x}$
\vfill
\item Find\marginpar{\small(6)} 
  the second derivative $d^2y/dx^2$ for the implicitly defined function 
  $\ds x^3+y^3=1$.
\vfill
\newpage
%%%%%%%%%%%%%%%%%%%%%%%%%%%%%%%%%%%%%%%%%%%%%%%%%%%%%%%%%%%%%% begin page 4 %%%
\item Two\marginpar{\small(5)} 
  cars start moving from the same point.  One travels 
  north at $80$ km/h and the other travels west at $60$ km/h.
  At what rate is the distance between the cars increasing two hours later?
\vfill
\newpage
%%%%%%%%%%%%%%%%%%%%%%%%%%%%%%%%%%%%%%%%%%%%%%%%%%%%%%%%%%%%%% begin page 5 %%%
\item When\marginpar{\small(5)} 
  blood flows along a blood vessel, the flux $F$ (the volume of blood per
  unit time that flows past a given point) is proportional to the fourth
  power of the radius of the blood vessel:
  \begin{equation*}
    F = k R^4
  \end{equation*}
  for some constant $k$.  Find the differential $dF$ and 
  use it to show that the \textbf{relative}
  change in $F$, namely $dF/F$,
  is about four times the relative change in $R$, namely $dR/R$.  How would
  a 5\% increase in the radius affect the flow of blood?
\newpage
%%%%%%%%%%%%%%%%%%%%%%%%%%%%%%%%%%%%%%%%%%%%%%%%%%%%%%%%%%%%%% begin page 6 %%%
\item Find\marginpar{\small(5)} 
  $\ds \lim_{\theta\to 0} \frac{\sin\theta}{\theta+\tan\theta}$.  
\vfill
\end{enumerate}

\end{document}

