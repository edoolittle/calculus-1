\documentclass{article}
\usepackage{amsmath}
\usepackage{fullpage}
%\usepackage{graphicx}
\usepackage{tikz}
\usetikzlibrary{calc}
% \usepackage{pgfmath}
\usepackage[aux]{rerunfilecheck}

% Macros for MATH 110 course dates

\newcommand{\commonTheme}{metropolis}
\newcommand{\commonColorTheme}{metropolis}

\newcommand{\commonAuthor}{Edward Doolittle}
\newcommand{\commonInstitute}{Department of Indigenous Knowledge and
  Science \\ First Nations University of Canada}
\newcommand{\commonCourse}{MATH 110 Calculus I}
\newcommand{\commonTerm}{202510}
\newcommand{\commonDate}{January 6, 2025}

% Review Material

% Lab 0
\newcommand{\commonEventNegativeOne}{LabNegativeOne}
\newcommand{\commonDateLabNegativeOne}{Monday, January 6, 2025}
\newcommand{\commonTitleLabNegativeOne}{MATH 110 Lab 0}
\newcommand{\commonSubtitleLabNegativeOne}{No Lab; Course Opens}

% Section 001
\newcommand{\commonEventZeroZeroOne}{ZeroZeroOne}
\newcommand{\commonDateZeroZeroOne}{Tuesday, January 7, 2025}
\newcommand{\commonTitleZeroZeroOne}{MATH 110 Review 0.1}
\newcommand{\commonSubtitleZeroZeroOne}{Review of Algebra}
\newcommand{\commonPSTitleZeroZeroOne}{MATH 110 Review Problem Set 0.1}

% Section 00A
\newcommand{\commonEventZeroZeroA}{ZeroZeroA}
\newcommand{\commonDateZeroZeroA}{Tuesday, January 7, 2025}
\newcommand{\commonTitleZeroZeroA}{MATH 110 Review 0.A}
\newcommand{\commonSubtitleZeroZeroA}{Review of Inequalities and
  Absolute Values}
\newcommand{\commonPSTitleZeroZeroA}{MATH 110 Review Problem Set 0.A}

% Section 00B
\newcommand{\commonEventZeroZeroB}{ZeroZeroB}
\newcommand{\commonDateZeroZeroB}{Tuesday, January 7, 2025}
\newcommand{\commonTitleZeroZeroB}{MATH 110 Review 0.B}
\newcommand{\commonSubtitleZeroZeroB}{Review of Coordinate Geometry
  and Lines}
\newcommand{\commonPSTitleZeroZeroB}{MATH 110 Review Problem Set 0.B}

% Section 00C
\newcommand{\commonEventZeroZeroC}{ZeroZeroC}
\newcommand{\commonDateZeroZeroC}{Thursday, January 9, 2025}
\newcommand{\commonTitleZeroZeroC}{MATH 110 Review 0.C}
\newcommand{\commonSubtitleZeroZeroC}{Review of Graphs of Second
  Degree Equations}
\newcommand{\commonPSTitleZeroZeroC}{MATH 110 Review Problem Set 0.C}

% Section 00D
\newcommand{\commonEventZeroZeroD}{ZeroZeroD}
\newcommand{\commonDateZeroZeroD}{Thursday, January 9, 2025}
\newcommand{\commonTitleZeroZeroD}{MATH 110 Review 0.D}
\newcommand{\commonSubtitleZeroZeroD}{Review of Trigonometry}
\newcommand{\commonPSTitleZeroZeroD}{MATH 110 Review Problem Set 0.D}

% Section 011
\newcommand{\commonEventZeroOneOne}{ZeroOneOne}
\newcommand{\commonDateZeroOneOne}{Thursday, January 9, 2025}
\newcommand{\commonTitleZeroOneOne}{MATH 110 Review 1.1}
\newcommand{\commonSubtitleZeroOneOne}{Review of Functions}
\newcommand{\commonPSTitleZeroOneOne}{MATH 110 Review Problem Set 1.1}


% Main Course

% Lab 1
\newcommand{\commonEventZero}{LabZero}
\newcommand{\commonDateLabZero}{Monday, January 13, 2025}
\newcommand{\commonTitleLabZero}{MATH 110 Lab 1}
\newcommand{\commonSubtitleLabZero}{Quiz 0: STACK, Onboarding}

% Section 1.4
\newcommand{\commonEventOne}{ZeroOneFour}
\newcommand{\commonDateZeroOneFour}{Tuesday, January 14, 2025}
\newcommand{\commonTitleZeroOneFour}{MATH 110 Lecture 1.4}
\newcommand{\commonSubtitleZeroOneFour}{The Tangent and Velocity Problems}
\newcommand{\commonPSTitleZeroOneFour}{MATH 110 Problem Set 1.4}

% Section 1.5
\newcommand{\commonEventTwo}{ZeroOneFive}
\newcommand{\commonDateZeroOneFive}{Thursday, January 16, 2025}
\newcommand{\commonTitleZeroOneFive}{MATH 110 Lecture 1.5}
\newcommand{\commonSubtitleZeroOneFive}{The Limit of a Function}
\newcommand{\commonPSTitleZeroOneFive}{MATH 110 Problem Set 1.5}

% Lab 2
\newcommand{\commonEventThree}{LabOne}
\newcommand{\commonDateLabOne}{Monday, January 20, 2025}
\newcommand{\commonTitleLabOne}{MATH 110 Lab 2}
\newcommand{\commonSubtitleLabOne}{Quiz 1: Review}

% Section 1.6
\newcommand{\commonEventFour}{ZeroOneSix}
\newcommand{\commonDateZeroOneSix}{Tuesday, January 21, 2025}
\newcommand{\commonTitleZeroOneSix}{MATH 110 Lecture 1.6}
\newcommand{\commonSubtitleZeroOneSix}{Calculating Limits Using the Limit Laws}
\newcommand{\commonPSTitleZeroOneSix}{MATH 110 Problem Set 1.6}

% Section 1.7
\newcommand{\commonEventFive}{ZeroOneSeven}
\newcommand{\commonDateZeroOneSeven}{(Not covered)}
\newcommand{\commonTitleZeroOneSeven}{MATH 110 Lecture 1.7}
\newcommand{\commonSubtitleZeroOneSeven}{The Precise Definition of a Limit}
\newcommand{\commonPSTitleZeroOneSeven}{MATH 110 Problem Set 1.7}

% Section 1.8
\newcommand{\commonEventSix}{ZeroOneEight}
\newcommand{\commonDateZeroOneEight}{Thursday, January 23, 2025}
\newcommand{\commonTitleZeroOneEight}{MATH 110 Lecture 1.8}
\newcommand{\commonSubtitleZeroOneEight}{Continuity}
\newcommand{\commonPSTitleZeroOneEight}{MATH 110 Problem Set 1.8}

% Lab 3
\newcommand{\commonEventSeven}{LabTwo}
\newcommand{\commonDateLabTwo}{Monday, January 27, 2025}
\newcommand{\commonTitleLabTwo}{MATH 110 Lab 3}
\newcommand{\commonSubtitleLabTwo}{Quiz 2: Sections 1.4, 1.5}

% Section 2.1
\newcommand{\commonEventEight}{ZeroTwoOne}
\newcommand{\commonDateZeroTwoOne}{Tuesday, January 28, 2025}
\newcommand{\commonTitleZeroTwoOne}{MATH 110 Lecture 2.1}
\newcommand{\commonSubtitleZeroTwoOne}{Derivatives and Rates of Change}
\newcommand{\commonPSTitleZeroTwoOne}{MATH 110 Problem Set 2.1}

% Section 2.2
\newcommand{\commonEventNine}{ZeroTwoTwo}
\newcommand{\commonDateZeroTwoTwo}{Thursday, January 30, 2025}
\newcommand{\commonTitleZeroTwoTwo}{MATH 110 Lecture 2.2}
\newcommand{\commonSubtitleZeroTwoTwo}{The Derivative as a Function}
\newcommand{\commonPSTitleZeroTwoTwo}{MATH 110 Problem Set 2.2}

% Lab 4
\newcommand{\commonEventTen}{LabThree}
\newcommand{\commonDateMTOne}{Monday, February 3, 2025} 
\newcommand{\commonDateLabThree}{Monday, February 3, 2025}
\newcommand{\commonTitleLabThree}{MATH 110 Lab 4}
\newcommand{\commonSubtitleLabThree}{Midterm: Review, Chapter 1}

% Section 2.3
\newcommand{\commonEventEleven}{ZeroTwoThree}
\newcommand{\commonDateZeroTwoThree}{Tuesday, February 4, 2025}
\newcommand{\commonTitleZeroTwoThree}{MATH 110 Lecture 2.3}
\newcommand{\commonSubtitleZeroTwoThree}{Differentiation Formulas}
\newcommand{\commonPSTitleZeroTwoThree}{MATH 110 Problem Set 2.3}

% Section 2.4
\newcommand{\commonEventTwelve}{ZeroTwoFour}
\newcommand{\commonDateZeroTwoFour}{Thursday, February 6, 2025}
\newcommand{\commonTitleZeroTwoFour}{MATH 110 Lecture 2.4}
\newcommand{\commonSubtitleZeroTwoFour}{Derivatives of Trigonometric Functions}
\newcommand{\commonPSTitleZeroTwoFour}{MATH 110 Problem Set 2.4}

% Lab 5
\newcommand{\commonEventThirteen}{LabFour}
\newcommand{\commonDateLabFour}{Monday, February 10, 2025}
\newcommand{\commonTitleLabFour}{MATH 110 Lab 5}
\newcommand{\commonSubtitleLabFour}{Quiz 3: Sections 2.1, 2.2}

% Section 2.5
\newcommand{\commonEventFourteen}{ZeroTwoFive}
\newcommand{\commonDateZeroTwoFive}{Tuesday, February 11, 2025}
\newcommand{\commonTitleZeroTwoFive}{MATH 110 Lecture 2.5}
\newcommand{\commonSubtitleZeroTwoFive}{The Chain Rule}
\newcommand{\commonPSTitleZeroTwoFive}{MATH 110 Problem Set 2.5}

% Section 2.6
\newcommand{\commonEventFifteen}{ZeroTwoSix}
\newcommand{\commonDateZeroTwoSix}{Thursday, February 13, 2025}
\newcommand{\commonTitleZeroTwoSix}{MATH 110 Lecture 2.6}
\newcommand{\commonSubtitleZeroTwoSix}{Implicit Differentiation}
\newcommand{\commonPSTitleZeroTwoSix}{MATH 110 Problem Set 2.6}

% Lab 6
\newcommand{\commonEventSixteen}{LabFive}
\newcommand{\commonDateLabFive}{Monday, February 24, 2025}
\newcommand{\commonTitleLabFive}{MATH 110 Lab 6}
\newcommand{\commonSubtitleLabFive}{Quiz 4: Sections 2.3, 2.4}

% Section 2.7
\newcommand{\commonEventSeventeen}{ZeroTwoSeven}
\newcommand{\commonDateZeroTwoSeven}{Tuesday, February 25, 2025}
\newcommand{\commonTitleZeroTwoSeven}{MATH 110 Lecture 2.7}
\newcommand{\commonSubtitleZeroTwoSeven}{Rates of Change in the
  Natural and Social Sciences}
\newcommand{\commonPSTitleZeroTwoSeven}{MATH 110 Problem Set 2.7}

% Section 2.8
\newcommand{\commonEventEighteen}{ZeroTwoEight}
\newcommand{\commonDateZeroTwoEight}{Thursday, February 27, 2025}
\newcommand{\commonTitleZeroTwoEight}{MATH 110 Lecture 2.8}
\newcommand{\commonSubtitleZeroTwoEight}{Related Rates}
\newcommand{\commonPSTitleZeroTwoEight}{MATH 110 Problem Set 2.8}

% Lab 7
\newcommand{\commonEventNineteen}{LabSix}
\newcommand{\commonDateLabSix}{Monday, March 3, 2025}
\newcommand{\commonTitleLabSix}{MATH 110 Lab 7}
\newcommand{\commonSubtitleLabSix}{Quiz 5: Sections 2.5, 2.6}

% Section 3.1
\newcommand{\commonEventTwenty}{ZeroThreeOne}
\newcommand{\commonDateZeroThreeOne}{Tuesday, March 4, 2025}
\newcommand{\commonTitleZeroThreeOne}{MATH 110 Lecture 3.1}
\newcommand{\commonSubtitleZeroThreeOne}{Maximum and Minimum Values}
\newcommand{\commonPSTitleZeroThreeOne}{MATH 11 Problem Set 3.1}

% Section 3.2
\newcommand{\commonEventTwentyOne}{ZeroThreeTwo}
\newcommand{\commonDateZeroThreeTwo}{Thursday, March 6, 2025}
\newcommand{\commonTitleZeroThreeTwo}{MATH 110 Lecture 3.2}
\newcommand{\commonSubtitleZeroThreeTwo}{The Mean Value Theorem}
\newcommand{\commonPSTitleZeroThreeTwo}{MATH 110 Problem Set 3.2}

% Lab 8
\newcommand{\commonEventTwentyTwo}{LabSeven}
\newcommand{\commonDateMTTwo}{Monday, March 10, 2025}
\newcommand{\commonDateLabSeven}{Monday, March 10, 2025}
\newcommand{\commonTitleLabSeven}{MATH 110 Lab 8}
\newcommand{\commonSubtitleLabSeven}{Midterm: Chapter 2}

% Section 3.3
\newcommand{\commonEventTwentyThree}{ZeroThreeThree}
\newcommand{\commonDateZeroThreeThree}{Tuesday, March 11, 2025}
\newcommand{\commonTitleZeroThreeThree}{MATH 110 Lecture 3.3}
\newcommand{\commonSubtitleZeroThreeThree}{How Derivatives Affect the
  Shape of a Graph}
\newcommand{\commonPSTitleZeroThreeThree}{MATH 110 Problem Set 3.3}

% Section 3.4
\newcommand{\commonEventTwentyFour}{ZeroThreeFour}
\newcommand{\commonDateZeroThreeFour}{Thursday, March 13, 2025}
\newcommand{\commonTitleZeroThreeFour}{MATH 110 Lecture 3.4}
\newcommand{\commonSubtitleZeroThreeFour}{Limits at Infinity;
  Horizontal Asymptotes}
\newcommand{\commonPSTitleZeroThreeFour}{MATH 110 Problem Set 3.4}

% Lab 9
\newcommand{\commonEventTwentyFive}{LabEight}
\newcommand{\commonDateLabEight}{Monday, March 17, 2025}
\newcommand{\commonTitleLabEight}{MATH 110 Lab 9}
\newcommand{\commonSubtitleLabEight}{Quiz 6: Sections 3.1, 3.2}

% Section 3.5
\newcommand{\commonEventTwentySix}{ZeroThreeFive}
\newcommand{\commonDateZeroThreeFive}{Tuesday, March 18, 2025}
\newcommand{\commonTitleZeroThreeFive}{MATH 110 Lecture 3.5}
\newcommand{\commonSubtitleZeroThreeFive}{Summary of Curve Sketching}
\newcommand{\commonPSTitleZeroThreeFive}{MATH 110 Problem Set 3.5}

% Section 3.7
\newcommand{\commonEventTwentySeven}{ZeroThreeSeven}
\newcommand{\commonDateZeroThreeSeven}{Thursday, March 20, 2025}
\newcommand{\commonTitleZeroThreeSeven}{MATH 110 Lecture 3.7}
\newcommand{\commonSubtitleZeroThreeSeven}{Optimization Problems}
\newcommand{\commonPSTitleZeroThreeSeven}{MATH 110 Problem Set 3.7}

% Lab 10
\newcommand{\commonEventTwentyEight}{LabNine}
\newcommand{\commonDateLabNine}{Monday, March 24, 2025}
\newcommand{\commonTitleLabNine}{MATH 110 Lab 10}
\newcommand{\commonSubtitleLabNine}{Quiz 7: Sections 3.3, 3.4}

% Section 4.1
\newcommand{\commonEventTwentyNine}{ZeroFourOne}
\newcommand{\commonDateZeroFourOne}{Tuesday, March 25, 2025}
\newcommand{\commonTitleZeroFourOne}{MATH 110 Lecture 4.1}
\newcommand{\commonSubtitleZeroFourOne}{Areas and Distances}
\newcommand{\commonPSTitleZeroFourOne}{MATH 110 Problem Set 4.1}

% Section 4.2
\newcommand{\commonEventThirty}{ZeroFourTwo}
\newcommand{\commonDateZeroFourTwo}{Thursday, March 27, 2025}
\newcommand{\commonTitleZeroFourTwo}{MATH 110 Lecture 4.2}
\newcommand{\commonSubtitleZeroFourTwo}{The Definite Integral}
\newcommand{\commonPSTitleZeroFourTwo}{MATH 110 Problem Set 4.2}

% Lab 11
\newcommand{\commonEventThirtyOne}{LabTen}
\newcommand{\commonDateLabTen}{Monday, March 31, 2025}
\newcommand{\commonTitleLabTen}{MATH 110 Lab 11}
\newcommand{\commonSubtitleLabTen}{Quiz 8: Sections 3.5, 3.7}

% Section 4.3
\newcommand{\commonEventThirtyTwo}{ZeroFourThree}
\newcommand{\commonDateZeroFourThree}{Tuesday, April 1, 2025}
\newcommand{\commonTitleZeroFourThree}{MATH 110 Lecture 4.3}
\newcommand{\commonSubtitleZeroFourThree}{The Fundamental Theorem of Calculus}
\newcommand{\commonPSTitleZeroFourThree}{MATH 110 Problem Set 4.3}

% Section 4.4
\newcommand{\commonEventThirtyThree}{ZeroFourFour}
\newcommand{\commonDateZeroFourFour}{Thursday, April 3, 2025}
\newcommand{\commonTitleZeroFourFour}{MATH 110 Lecture 4.4}
\newcommand{\commonSubtitleZeroFourFour}{Indefinite Integrals and the
  Net Change Theorem}
\newcommand{\commonPSTitleZeroFourFour}{MATH 110 Problem Set 4.4}

% Lab 12
\newcommand{\commonEventThirtyFour}{LabEleven}
\newcommand{\commonDateLabEleven}{Monday, April 7, 2025}
\newcommand{\commonTitleLabEleven}{MATH 110 Lab 12}
\newcommand{\commonSubtitleLabEleven}{Quiz 9: Sections 4.1, 4.2}

% Section 4.5
\newcommand{\commonEventThirtyFive}{ZeroFourFive}
\newcommand{\commonDateZeroFourFive}{Tuesday, April 8, 2025}
\newcommand{\commonTitleZeroFourFive}{MATH 110 Lecture 4.5}
\newcommand{\commonSubtitleZeroFourFive}{The Substitution Rule}
\newcommand{\commonPSTitleZeroFourFive}{MATH 110 Problem Set 4.5}

% Section 5.1
\newcommand{\commonEventThirtySix}{ZeroFiveOne}
\newcommand{\commonDateZeroFiveOne}{Thursday, April 10, 2025}
\newcommand{\commonTitleZeroFiveOne}{MATH 110 Lecture 5.1}
\newcommand{\commonSubtitleZeroFiveOne}{Areas Between Curves}
\newcommand{\commonPSTitleZeroFiveOne}{MATH 110 Problem Set 5.1}

% Lab 13
\newcommand{\commonEventThirtySeven}{LabTwelve}
\newcommand{\commonDateLabTwelve}{Monday, April 14, 2025}
\newcommand{\commonTitleLabTwelve}{MATH 110 Review Lab}
\newcommand{\commonSubtitleLabTwelve}{Bonus Quiz 10: Sections 4.3, 4.4}

% Final Class
\newcommand{\commonEventThirtyEight}{FinalClass}
\newcommand{\commonDateFinalClass}{Tuesday, April 15, 2025}
\newcommand{\commonTitleFinalClass}{MATH 110 Review Class}
\newcommand{\commonSubtitleFinalClass}{Answer Questions, Review for Exam}

% Final Exam
\newcommand{\commonEventThirtyNine}{Final}
\newcommand{\commonDateFinal}{Thursday, April 22, 2025}
\newcommand{\commonTitleFinal}{MATH 110 Final Exam}
\newcommand{\commonSubtitleFinal}{Comprehensive Exam: All Sections}

% Orphaned -- no longer part of the course

% Section 2.9
\newcommand{\commonDateZeroTwoNine}{Not part of the course}
\newcommand{\commonTitleZeroTwoNine}{MATH 110 Lecture 2.9}
\newcommand{\commonSubtitleZeroTwoNine}{Linear Approximations and Differentials}
\newcommand{\commonPSTitleZeroTwoNine}{MATH 110 Problem Set 2.9}


% % Introduction
% \newcommand{\commonEventOneDate}{Wednesday, September 8, 2010}
% \newcommand{\commonEventOneDesc}{Introduction to the Course}
% \newcommand{\commonDateZeroZeroZero}{September 8, 2010}
% \newcommand{\commonTitleZeroZeroZero}{MATH 104 Introduction}
% \newcommand{\commonSubtitleZeroZeroZero}{Outline of the Course}

% % Lecture 1
% \newcommand{\commonEventTwoDate}{Friday, September 10, 2010}
% \newcommand{\commonEventTwoDesc}{Lecture 1: Algebra}
% \newcommand{\commonDateZeroZeroOne}{September 10, 2010}
% \newcommand{\commonTitleZeroZeroOne}{MATH 104 Lecture 1}
% \newcommand{\commonSubtitleZeroZeroOne}{Review of Algebra}
% % associated evaluation ... factor this out?
% \newcommand{\commonPSTitleZeroZeroOne}{MATH 104 Problem Set 1}
% \newcommand{\commonEvalZeroZeroOne}{Quiz 1}
% \newcommand{\commonEvalDateZeroZeroOne}{Wednesday, September 15, 2010}

% % Lecture 2
% \newcommand{\commonEventThreeDate}{Monday, September 13, 2010}
% \newcommand{\commonEventThreeDesc}{Lecture 2: Appendix A}
% \newcommand{\commonDateZeroZeroA}{September 13, 2010}
% \newcommand{\commonTitleZeroZeroA}{MATH 104 Lecture 2}
% \newcommand{\commonSubtitleZeroZeroA}{Appendix A: Numbers, Inequalities, 
%   and Absolute Values}
% % associated evaluation ... factor this out?
% \newcommand{\commonPSTitleZeroZeroA}{MATH 104 Problem Set 2}
% \newcommand{\commonEvalZeroZeroA}{Quiz 2}
% \newcommand{\commonEvalDateZeroZeroA}{Wednesday, September 22, 2010}

% % Review 1
% \newcommand{\commonEventFourDate}{Wednesday, September 15, 2010}
% \newcommand{\commonEventFourDesc}{Review 1: Review Algebra; Quiz 1; Review Appendix A}
% \newcommand{\commonDateRZeroOne}{September 15, 2010}
% \newcommand{\commonTitleRZeroOne}{MATH 104 Review 1}
% \newcommand{\commonSubtitleRZeroOne}{Review of Algebra, Appendix A}

% % Lecture 3
% \newcommand{\commonEventFiveDate}{Friday, September 17, 2010}
% \newcommand{\commonEventFiveDesc}{Lecture 3: Appendix B}
% \newcommand{\commonDateZeroZeroB}{September 17, 2010}
% \newcommand{\commonTitleZeroZeroB}{MATH 104 Lecture 3}
% \newcommand{\commonSubtitleZeroZeroB}{Appendix B: Coordinate Geometry and Lines}
% % associated evaluation ... factor this out?
% \newcommand{\commonPSTitleZeroZeroB}{MATH 104 Problem Set 3}
% \newcommand{\commonEvalZeroZeroB}{Quiz 2}
% \newcommand{\commonEvalDateZeroZeroB}{Wednesday, September 22, 2010}

% % Lecture 4
% \newcommand{\commonEventSixDate}{Monday, Sepbember 20, 2010}
% \newcommand{\commonEventSixDesc}{Lecture 4: Appendix C}
% \newcommand{\commonDateZeroZeroC}{September 20, 2010}
% \newcommand{\commonTitleZeroZeroC}{MATH 104 Lecture 4}
% \newcommand{\commonSubtitleZeroZeroC}{Appendix C: Graphs of Second-Degree Equations}
% % associated evaluation ... factor this out?
% \newcommand{\commonPSTitleZeroZeroC}{MATH 104 Problem Set 4}
% \newcommand{\commonEvalZeroZeroC}{Midterm 0}
% \newcommand{\commonEvalDateZeroZeroC}{Wednesday, September 29, 2010}

% % Review 2
% \newcommand{\commonEventSevenDate}{Wednesday, September 22, 2010}
% \newcommand{\commonEventSevenDesc}{Review 2: Review Appendix B; Quiz 2; Review Appendix C}
% \newcommand{\commonDateRZeroTwo}{September 22, 2010}
% \newcommand{\commonTitleRZeroTwo}{MATH 104 Review 2}
% \newcommand{\commonSubtitleRZeroTwo}{Review of Appendices B and C}

% % Lecture 5
% \newcommand{\commonEventEightDate}{Friday, September 24, 2010}
% \newcommand{\commonEventEightDesc}{Lecture 5: Appendix D}
% \newcommand{\commonDateZeroZeroD}{September 24, 2010}
% \newcommand{\commonTitleZeroZeroD}{MATH 104 Lecture 5}
% \newcommand{\commonSubtitleZeroZeroD}{Appendix D: Trigonometry}
% % associated evaluation ... factor this out?
% \newcommand{\commonPSTitleZeroZeroD}{MATH 104 Problem Set 5}
% \newcommand{\commonEvalZeroZeroD}{Midterm 0}
% \newcommand{\commonEvalDateZeroZeroD}{Wednesday, September 29, 2010}

% % Lecture 6
% \newcommand{\commonEventNineDate}{Monday, September 27, 2010}
% \newcommand{\commonEventNineDesc}{Lecture 6: Section 1.1}
% \newcommand{\commonDateZeroOneOne}{September 27, 2010}
% \newcommand{\commonTitleZeroOneOne}{MATH 104 Lecture 6}
% \newcommand{\commonSubtitleZeroOneOne}{Section 1.1: Four Ways to Represent a Function}
% % associated evaluation ... factor this out?
% \newcommand{\commonPSTitleZeroOneOne}{MATH 104 Problem Set 6}
% \newcommand{\commonEvalZeroOneOne}{Quiz 3}
% \newcommand{\commonEvalDateZeroOneOne}{Wednesday, October 6, 2010}

% % Review 3
% \newcommand{\commonEventTenDate}{Wednesday, September 29, 2010}
% \newcommand{\commonEventTenDesc}{Review 3: Review Appendix D; 
%   Self-Assessment Midterm 0}
% \newcommand{\commonDateRZeroThree}{September 29, 2010}
% \newcommand{\commonTitleRZeroThree}{MATH 104 Review 3}
% \newcommand{\commonSubtitleRZeroThree}{Review of Appendix D}

% % Lecture 7
% \newcommand{\commonEventElevenDate}{Friday, October 1, 2010}
% \newcommand{\commonEventElevenDesc}{Lecture 7: Section 1.2}
% \newcommand{\commonDateZeroOneTwo}{October 1, 2010}
% \newcommand{\commonTitleZeroOneTwo}{MATH 104 Lecture 7}
% \newcommand{\commonSubtitleZeroOneTwo}{Section 1.2: Mathematical Models: A Catalog of Essential Functions}
% % associated evaluation ... factor this out?
% \newcommand{\commonPSTitleZeroOneTwo}{MATH 104 Problem Set 7}
% \newcommand{\commonEvalZeroOneTwo}{Quiz 3}
% \newcommand{\commonEvalDateZeroOneTwo}{Wednesday, October 6, 2010}

% % Lecture 8
% \newcommand{\commonEventTwelveDate}{Monday, October 4, 2010}
% \newcommand{\commonEventTwelveDesc}{Lecture 8: Section 1.3}
% \newcommand{\commonDateZeroOneThree}{October 4, 2010}
% \newcommand{\commonTitleZeroOneThree}{MATH 104 Lecture 8}
% \newcommand{\commonSubtitleZeroOneThree}{Section 1.3: New Functions from Old Functions}
% % associated evaluation ... factor this out?
% \newcommand{\commonPSTitleZeroOneThree}{MATH 104 Problem Set 8}
% \newcommand{\commonEvalZeroOneThree}{Quiz 4}
% \newcommand{\commonEvalDateZeroOneThree}{Wednesday, October 13, 2010}

% % Review 4
% \newcommand{\commonEventThirteenDate}{Wednesday, October 6, 2010}
% \newcommand{\commonEventThirteenDesc}{Review 4: Review 1.1, 1.2; Quiz 3}
% \newcommand{\commonDateROneOne}{October 6, 2010}
% \newcommand{\commonTitleROneOne}{MATH 104 Review 4}
% \newcommand{\commonSubtitleROneOne}{Reveiw of 1.1, 1.2}

% % Lecture 9
% \newcommand{\commonEventFourteenDate}{Friday, October 8, 2010}
% \newcommand{\commonEventFourteenDesc}{Lecture 9: Section 1.4}
% \newcommand{\commonDateZeroOneFour}{October 8, 2010}
% \newcommand{\commonTitleZeroOneFour}{MATH 104 Lecture 9}
% \newcommand{\commonSubtitleZeroOneFour}{Section 1.4: Graphing Calculators and Computers}
% % associated evaluation ... factor this out?
% \newcommand{\commonPSTitleZeroOneFour}{MATH 104 Problem Set 9}
% \newcommand{\commonEvalZeroOneFour}{Quiz 4}
% \newcommand{\commonEvalDateZeroOneFour}{Wednesday, October 13, 2010}

% % Thanksgiving holiday
% \newcommand{\commonEventFifteenDate}{Monday, October 11, 2010}
% \newcommand{\commonEventFifteenDesc}{No class: Thanksgiving holiday}

% % Review 5
% \newcommand{\commonEventSixteenDate}{Wednesday, October 13, 2010}
% \newcommand{\commonEventSixteenDesc}{Review 5: Review 1.3, 1.4; Quiz 4}
% \newcommand{\commonDateROneTwo}{October 13, 2010}
% \newcommand{\commonTitleROneTwo}{MATH 104 Review 5}
% \newcommand{\commonSubtitleOneRTwo}{Review of 1.3, 1.4}

% % Lecture 10
% \newcommand{\commonEventSeventeenDate}{Friday, October 15, 2010}
% \newcommand{\commonEventSeventeenDesc}{Lecture 10: Section 1.5}
% \newcommand{\commonDateZeroOneFive}{October 15, 2010}
% \newcommand{\commonTitleZeroOneFive}{MATH 104 Lecture 10}
% \newcommand{\commonSubtitleZeroOneFive}{Section 1.5: Exponential Functions}
% % associated evaluation ... factor this out?
% \newcommand{\commonPSTitleZeroOneFive}{MATH 104 Problem Set 10}
% \newcommand{\commonEvalZeroOneFive}{Quiz 5}
% \newcommand{\commonEvalDateZeroOneFive}{Wednesday, October 20, 2010}

% % Lecture 11
% \newcommand{\commonEventEighteenDate}{Monday, October 18, 2010}
% \newcommand{\commonEventEighteenDesc}{Lecture 11: Section 1.6}
% \newcommand{\commonDateZeroOneSix}{October 18, 2010}
% \newcommand{\commonTitleZeroOneSix}{MATH 104 Lecture 11}
% \newcommand{\commonSubtitleZeroOneSix}{Section 1.6: Inverse Functions and Logarithms}
% % associated evaluation ... factor this out?
% \newcommand{\commonPSTitleZeroOneSix}{MATH 104 Problem Set 11}
% \newcommand{\commonEvalZeroOneSix}{Midterm 1}
% \newcommand{\commonEvalDateZeroOneSix}{Wednesday, October 27, 2010}

% % Review 6
% \newcommand{\commonEventNineteenDate}{Wednesday, October 20, 2010}
% \newcommand{\commonEventNineteenDesc}{Review 6: Review 1.5; Quiz 5; Review 1.6}
% \newcommand{\commonDateROneThree}{October 20, 2010}
% \newcommand{\commonDateZeroOneR}{October 20, 2010}
% \newcommand{\commonTitleROneThree}{MATH 104 Review 6}
% \newcommand{\commonSubtitleROneThree}{Review of 1.5, 1.6}
% % associated evaluation ... factor this out?
% \newcommand{\commonPSTitleZeroOneR}{MATH 104 Problem Set R1}
% \newcommand{\commonEvalZeroOneR}{Midterm 1}
% \newcommand{\commonEvalDateZeroOneR}{Wednesday, October 27, 2010}

% % Lecture 12
% \newcommand{\commonEventTwentyDate}{Friday, October 22, 2010}
% \newcommand{\commonEventTwentyDesc}{Lecture 12: Section 2.1}
% \newcommand{\commonDateZeroTwoOne}{October 22, 2010}
% \newcommand{\commonTitleZeroTwoOne}{MATH 104 Lecture 12}
% \newcommand{\commonSubtitleZeroTwoOne}{Section 2.1: The Tangent and Velocity Problems}
% % associated evaluation ... factor this out?
% \newcommand{\commonPSTitleZeroTwoOne}{MATH 104 Problem Set 12}
% \newcommand{\commonEvalZeroTwoOne}{Quiz 6}
% \newcommand{\commonEvalDateZeroTwoOne}{Wednesday, November 3, 2010}

% % Lecture 13
% \newcommand{\commonEventTwentyOneDate}{Monday, October 25, 2010}
% \newcommand{\commonEventTwentyOneDesc}{Lecture 13: Section 2.2(a)}
% \newcommand{\commonDateZeroTwoTwoa}{October 25, 2010}
% \newcommand{\commonTitleZeroTwoTwoa}{MATH 104 Lecture 13}
% \newcommand{\commonSubtitleZeroTwoTwoa}{Section 2.2(a): The Limit of a Function I}
% % associated evaluation ... factor this out?
% \newcommand{\commonPSTitleZeroTwoTwoa}{MATH 104 Problem Set 13}
% \newcommand{\commonEvalZeroTwoTwoa}{Quiz 6}
% \newcommand{\commonEvalDateZeroTwoTwoa}{Wednesday, November 3, 2010}

% % Midterm Test 1
% % October 27, 2010
% \newcommand{\commonEventTwentyTwoDate}{Wednesday, October 27, 2010}
% \newcommand{\commonEventTwentyTwoDesc}{Midterm Test 1: Chapter 1}

% % Lecture 14
% \newcommand{\commonEventTwentyThreeDate}{Friday, October 29, 2010}
% \newcommand{\commonEventTwentyThreeDesc}{Lecture 14: Section 2.2(b)}
% \newcommand{\commonDateZeroTwoTwob}{October 29, 2010}
% \newcommand{\commonTitleZeroTwoTwob}{MATH 104 Lecture 14}
% \newcommand{\commonSubtitleZeroTwoTwob}{Section 2.2(b): The Limit of a Function II}
% % associated evaluation ... factor this out?
% \newcommand{\commonPSTitleZeroTwoTwob}{MATH 104 Problem Set 14}
% \newcommand{\commonEvalZeroTwoTwob}{Quiz 6}
% \newcommand{\commonEvalDateZeroTwoTwob}{Wednesday, November 3, 2010}

% % Lecture 15
% \newcommand{\commonEventTwentyFourDate}{Monday, November 1, 2010}
% \newcommand{\commonEventTwentyFourDesc}{Lecture 15: Section 2.3}
% \newcommand{\commonDateZeroTwoThree}{November 1, 2010}
% \newcommand{\commonTitleZeroTwoThree}{MATH 104 Lecture 15}
% \newcommand{\commonSubtitleZeroTwoThree}{Section 2.3: Calculating Limits Using the Limit Laws}
% % associated evaluation ... factor this out?
% \newcommand{\commonPSTitleZeroTwoThree}{MATH 104 Problem Set 15}
% \newcommand{\commonEvalZeroTwoThree}{Quiz 7}
% \newcommand{\commonEvalDateZeroTwoThree}{Wednesday, November 10, 2010}

% % Review 7
% \newcommand{\commonEventTwentyFiveDate}{Wednesday, November 3, 2010}
% \newcommand{\commonEventTwentyFiveDesc}{Review 7: Review 2.1, 2.2; Quiz 6; Review 2.3}
% \newcommand{\commonDateRTwoOne}{November 3, 2010}
% \newcommand{\commonTitleRTwoOne}{MATH 104 Review 7}
% \newcommand{\commonSubtitleRTwoOne}{Review of 2.1, 2.2, 2.3}

% % Lecture 16
% \newcommand{\commonEventTwentySixDate}{Friday, November 5, 2010}
% \newcommand{\commonEventTwentySixDesc}{Lecture 16: Section 2.5}
% \newcommand{\commonDateZeroTwoFive}{November 5, 2010}
% \newcommand{\commonTitleZeroTwoFive}{MATH 104 Lecture 16}
% \newcommand{\commonSubtitleZeroTwoFive}{Section 2.5: Continuity}
% % associated evaluation ... factor this out?
% \newcommand{\commonPSTitleZeroTwoFive}{MATH 104 Problem Set 16}
% \newcommand{\commonEvalZeroTwoFive}{Quiz 7}
% \newcommand{\commonEvalDateZeroTwoFive}{Wednesday, November 10, 2010}

% % Lecture 17
% \newcommand{\commonEventTwentySevenDate}{Monday, November 8, 2010}
% \newcommand{\commonEventTwentySevenDesc}{Lecture 17: Section 2.6}
% \newcommand{\commonDateZeroTwoSix}{November 8, 2010}
% \newcommand{\commonTitleZeroTwoSix}{MATH 104 Lecture 17}
% \newcommand{\commonSubtitleZeroTwoSix}{Section 2.6: Limits at Infinity: Horizontal Asymptotes}
% % associated evaluation ... factor this out?
% \newcommand{\commonPSTitleZeroTwoSix}{MATH 104 Problem Set 17}
% \newcommand{\commonEvalZeroTwoSix}{Quiz 8}
% \newcommand{\commonEvalDateZeroTwoSix}{Wednesday, November 17, 2010}

% % Review 8
% \newcommand{\commonEventTwentyEightDate}{Wednesday, November 10, 2010}
% \newcommand{\commonEventTwentyEightDesc}{Review 8: Review 2.5; Quiz 7; Review 2.6}
% \newcommand{\commonDateRTwoTwo}{November 10, 2010}
% \newcommand{\commonTitleRTwoTwo}{MATH 104 Review 8}
% \newcommand{\commonSubtitleRTwoTwo}{Review of 2.5, 2.6}

% % Lecture 18
% \newcommand{\commonEventTwentyNineDate}{Friday, November 12, 2010}
% \newcommand{\commonEventTwentyNineDesc}{Lecture 18: Section 2.7}
% \newcommand{\commonDateZeroTwoSeven}{November 12, 2010}
% \newcommand{\commonTitleZeroTwoSeven}{MATH 104 Lecture 18}
% \newcommand{\commonSubtitleZeroTwoSeven}{Section 2.7: Derivatives and Rates of Change}
% % associated evaluation ... factor this out?
% \newcommand{\commonPSTitleZeroTwoSeven}{MATH 104 Problem Set 18}
% \newcommand{\commonEvalZeroTwoSeven}{Quiz 8}
% \newcommand{\commonEvalDateZeroTwoSeven}{Wednesday, November 17, 2010}

% % Lecture 19
% \newcommand{\commonEventThirtyDate}{Monday, November 15, 2010}
% \newcommand{\commonEventThirtyDesc}{Lecture 19: Section 2.8}
% \newcommand{\commonDateZeroTwoEight}{November 15, 2010}
% \newcommand{\commonTitleZeroTwoEight}{MATH 104 Lecture 19}
% \newcommand{\commonSubtitleZeroTwoEight}{Section 2.8: The Derivative as a Function}
% % associated evaluation ... factor this out?
% \newcommand{\commonPSTitleZeroTwoEight}{MATH 104 Problem Set 19}
% \newcommand{\commonEvalZeroTwoEight}{Midterm 2}
% \newcommand{\commonEvalDateZeroTwoEight}{Wednesday, November 24, 2010}

% % Review 9
% % November 17, 2010
% \newcommand{\commonEventThirtyOneDate}{Wednesday, November 17, 2010}
% \newcommand{\commonEventThirtyOneDesc}{Review 9: Review 2.7; Quiz 8; Review 2.8}
% \newcommand{\commonDateRTwoThree}{November 17, 2010}
% \newcommand{\commonTitleRTwoThree}{MATH 104 Review 9}
% \newcommand{\commonSubtitleRTwoThree}{Review of 2.7, 2.8}

% % Lecture 20
% \newcommand{\commonEventThirtyTwoDate}{Friday, November 19, 2010}
% \newcommand{\commonEventThirtyTwoDesc}{Lecture 20: Section 3.1}
% \newcommand{\commonDateZeroThreeOne}{November 19, 2010}
% \newcommand{\commonTitleZeroThreeOne}{MATH 104 Lecture 20}
% \newcommand{\commonSubtitleZeroThreeOne}{Section 3.1: Derivatives of Polynomials and Exponential Functions}
% % associated evaluation ... factor this out?
% \newcommand{\commonPSTitleZeroThreeOne}{MATH 104 Problem Set 20}
% \newcommand{\commonEvalZeroThreeOne}{Quiz 9}
% \newcommand{\commonEvalDateZeroThreeOne}{Wednesday, December 1, 2010}

% % Lecture 21
% \newcommand{\commonEventThirtyThreeDate}{Monday, November 22, 2010}
% \newcommand{\commonEventThirtyThreeDesc}{Lecture 21: Section 3.2}
% \newcommand{\commonDateZeroThreeTwo}{November 22, 2010}
% \newcommand{\commonTitleZeroThreeTwo}{MATH 104 Lecture 21}
% \newcommand{\commonSubtitleZeroThreeTwo}{Section 3.2: The Product and Quotient Rules}
% % associated evaluation ... factor this out?
% \newcommand{\commonPSTitleZeroThreeTwo}{MATH 104 Problem Set 21}
% \newcommand{\commonEvalZeroThreeTwo}{Quiz 9}
% \newcommand{\commonEvalDateZeroThreeTwo}{Wednesday, December 1, 2010}

% % Midterm Test 2
% \newcommand{\commonEventThirtyFourDate}{Wednesday, November 24, 2010}
% \newcommand{\commonEventThirtyFourDesc}{Midterm Test 2: Chapter 2}

% % Lecture 22
% \newcommand{\commonEventThirtyFiveDate}{Friday, November 26, 2010}
% \newcommand{\commonEventThirtyFiveDesc}{Lecture 22: Section 3.3}
% \newcommand{\commonDateZeroThreeThree}{November 26, 2010}
% \newcommand{\commonTitleZeroThreeThree}{MATH 104 Lecture 22}
% \newcommand{\commonSubtitleZeroThreeThree}{Section 3.3: Derivatives of Trigonometric Functions}
% % associated evaluation ... factor this out?
% \newcommand{\commonPSTitleZeroThreeThree}{MATH 104 Problem Set 22}
% \newcommand{\commonEvalZeroThreeThree}{Quiz 9}
% \newcommand{\commonEvalDateZeroThreeThree}{Wednesday, December 1, 2010}

% % Lecture 23
% \newcommand{\commonEventThirtySixDate}{Monday, November 29, 2010}
% \newcommand{\commonEventThirtySixDesc}{Lecture 23: Section 3.4}
% \newcommand{\commonDateZeroThreeFour}{November 29, 2010}
% \newcommand{\commonTitleZeroThreeFour}{MATH 104 Lecture 23}
% \newcommand{\commonSubtitleZeroThreeFour}{Section 3.4: The Chain Rule}
% % associated evaluation ... factor this out?
% \newcommand{\commonPSTitleZeroThreeFour}{MATH 104 Problem Set 23}
% \newcommand{\commonEvalZeroThreeFour}{the final exam}
% \newcommand{\commonEvalDateZeroThreeFour}{Monday, December 13, 2010}

% % Review 10
% \newcommand{\commonEventThirtySevenDate}{Wednesday, December 1, 2010}
% \newcommand{\commonEventThirtySevenDesc}{Review 10: Review 3.1, 3.2, 3.3; Quiz 9}
% \newcommand{\commonDateRThreeTwo}{December 1, 2010}
% \newcommand{\commonTitleRThreeTwo}{MATH 104 Review 10}
% \newcommand{\commonSubtitleRThreeTwo}{Review of 3.1, 3.2, 3.3}

% % Lecture 24
% \newcommand{\commonEventThirtyEightDate}{Friday, December 3, 2010}
% \newcommand{\commonEventThirtyEightDesc}{Lecture 24: Section 3.5}
% \newcommand{\commonDateZeroThreeFive}{December 3, 2010}
% \newcommand{\commonTitleZeroThreeFive}{MATH 104 Lecture 24}
% \newcommand{\commonSubtitleZeroThreeFive}{Section 3.5: Implicit Differentiation}
% % associated evaluation ... factor this out?
% \newcommand{\commonPSTitleZeroThreeFive}{MATH 104 Problem Set 24}
% \newcommand{\commonEvalZeroThreeFive}{the final exam}
% \newcommand{\commonEvalDateZeroThreeFive}{Monday, December 13, 2010}

% % Lecture 25
% \newcommand{\commonEventThirtyNineDate}{Monday, December 6, 2010}
% \newcommand{\commonEventThirtyNineDesc}{Lecture 25: Section 3.6}
% \newcommand{\commonDateZeroThreeSix}{December 6, 2010}
% \newcommand{\commonTitleZeroThreeSix}{MATH 104 Lecture 25}
% \newcommand{\commonSubtitleZeroThreeSix}{Section 3.6: Derivatives of Logarithmic Functions}
% % associated evaluation ... factor this out?
% \newcommand{\commonPSTitleZeroThreeSix}{MATH 104 Problem Set 25}
% \newcommand{\commonEvalZeroThreeSix}{the final exam}
% \newcommand{\commonEvalDateZeroThreeSix}{Monday, December 13, 2010}

% % Review 11
% \newcommand{\commonEventFortyDate}{Wednesday, December 8, 2010}
% \newcommand{\commonEventFortyDesc}{(Bonus) Review 11: Review 3.4, 3.5, 3.6}
% \newcommand{\commonDateRThreeThree}{December 8, 2010}
% \newcommand{\commonTitleRThreeThree}{MATH 104 (Bonus) Review 11}
% \newcommand{\commonSubtitleRThreeThree}{Review of 3.4, 3.5, 3.6}

% % Final Exam
% % December 13, 2010
% \newcommand{\commonEventFinalDate}{Monday, December 13, 2010}
% \newcommand{\commonEventFinalDesc}{MATH 104 Final Exam}

%%% Local variables:
%%% mode: latex
%%% TeX-master: "MATH110-Syllabus.tex"
%%% End:

\newcommand{\ds}{\displaystyle}

\title{\commonPSTitleZeroZeroOne\ Solutions}
\author{\commonAuthor}
\date{\commonDateZeroZeroOne}

\begin{document}
\maketitle
\begin{enumerate}
\item %1 (Based on Algebra.1--12) Expand and simplify.
  \begin{enumerate}
  \item We have
    \begin{equation*}
      7(t^2-1)-2(t+3)-5t(t-4)
      = 7t^2 - 7 -2t -6 -5t^2+20t
      = 2t^2 +18t -13
    \end{equation*}
  \item By the binomial theorem and the distributive law,
    \begin{equation*}
      x(x-2)^2 = x(x^2-4x+4) = x^3 -4x^2 + 4x
    \end{equation*}
  \end{enumerate}
\item %2 (Based on Algebra.17--26) Simplify.
  \begin{enumerate}
  \item We put the fractions over a common denominator and add:
    \begin{equation*}
      \frac{1}{x+2} + \frac{1}{x-1}
      = \frac{x-1}{(x+2)(x-1)} + \frac{(x+2)}{(x+2)(x-1)}
      = \frac{2x+1}{(x+2)(x-1)}
    \end{equation*}
    You could expand the denominator, but experience tells me that
    it's best to leave it in factored form.
  \item Invert and multiply:
    \begin{equation*}
      \frac{x/a}{y/b} = \frac{x}{a} \div \frac{y}{b}
      = \frac{x}{a} \times \frac{b}{y} = \frac{bx}{ay}
    \end{equation*}
  \end{enumerate}
\item %3 (Based on Algebra.29--41) Factor the expression.
  \begin{enumerate}
  \item Each term has a factor of 5, and a factor of $a$, and a factor
    of $x$, so we take out a common factor of $5ax$ to obtain
    \begin{equation*}
      5ax - 35ax^2 = 5ax(1-7x)
    \end{equation*}
  \item\label{prob:quadAnonzero} % refer to this below
    Let's first try guessing a root.  A good place to start would be the
    factors of the constant term ($-3$), which has factors $1$, $-1$, $3$, and
    $-3$.  (This approach isn't guaranteed, but when it does, it saves
    us some work.)  Substituting in $x=1$ we get $2(1)^2-5(1)-3=-6\ne 0$, so
    $x=1$ is not a root.  Substituting in $x=-1$, we get $2(-1)^2-5(-1)-3=
    2+5-3=4\ne 0$, so $x=-1$ is not a root.  Substituting $x=3$ we get
    $2(3)^2-5(3)-3=18-15-3=0$, so $x=3$ is a root!

    Now we use the factor theorem to write
    \begin{equation*}
      2x^2-5x-3 = (x-3)(ax+b)
    \end{equation*}
    We don't know what $a$ and $b$ are, but we can find out by polynomial
    long division or short division.  I'll use short division.  We expand
    the RHS to obtain
    \begin{equation*}
      2x^2-5x-3 = ax^2 + bx - 3ax -3b = ax^2 + (b-3a) x -3b
    \end{equation*}
    comparing the coefficients of $x^2$, we see $2=a$.  Comparing the 
    constant terms, we see $-3=-3b$ which implies $b=1$.  Altogether we 
    should have
    \begin{equation*}
      2x^2-5x-3 = (x-3)(2x+1)
    \end{equation*}
    You should check that result by expanding the RHS.
  \end{enumerate}
\item %4 (Based on Algebra.49--51) Simplify the expression.
  \begin{enumerate}
  \item We factor the numerator and denominator, in search of common
    factors that can be canceled:
    \begin{equation*}
      \frac{x^2+2x-3}{x^2-9}
      = \frac{(x+3)(x-1)}{(x+3)(x-3)}
      = \frac{x-1}{x-3}
    \end{equation*}
  \item Again, we factor the numerator using the technique
    of question~\ref{prob:quadAnonzero}, which won't be reproduced here.
    The denominator is harder to factor, but note that for this problem
    to go anywhere, there has to be a factor of $x+2$ or a factor of 
    $5x-1$ in the denominator.  (If there isn't, then we can't simplify
    any further and we stop.) After some experimentation we have
    \begin{equation*}
      \frac{5x^2+9x-2}{10x^2-17x+3}
      = \frac{(x+2)(5x-1)}{(5x-1)(2x-3)} 
      = \frac{x+2}{2x-3}
    \end{equation*}
  \end{enumerate}
% EJD: Better solution: x^2+6x = x^2+6x+9 - 9
\item %5 (Based on Algebra.55--60) Complete the square.
  \begin{enumerate}
  \item We divide the coefficient of the $x$ term by $2$:
    \begin{equation*}
      x^2+6x-10 = (x+6/2)^2 + \mbox{junk}
      = x^2 + 6x + 9 + \mbox{junk}
    \end{equation*}
    After cancellation we see that ``junk'' has to be $-19$.  In summary,
    \begin{equation*}
      x^2+6x-10 = (x+3)^2 -19
    \end{equation*}
    You should check by expanding and gathering the RHS.
  \item Dealing with ``leading coefficients'' like the $3$ in this case is
    a pain in the neck.  See the textbook supplement 
    for one approach.  However, in
    this case I would use a slightly more creative approach.  Write
    \begin{equation*}
      3x^2-18x+7 = 3x^2 -18x + 6 + 1 = 3(x^2-6x+2) + 1
    \end{equation*}
    Now we just have to complete the square inside the bracket.  Similar
    to the previous problem we have
    \begin{equation*}
      x^2-6x+2 = (x-3)^2 + \mbox{junk} = x^2-6x+9 + \mbox{junk}
    \end{equation*}
    Cancelling and solving we have junk $=-7$.  Substituting that result
    into the previous, we have
    \begin{equation*}
      3x^2 - 18x + 7 = 3\left( (x-3)^2 - 7 \right) + 1
    \end{equation*}
    That's pretty good, but we can gather the constants all together:
    \begin{equation*}
      3x^2 - 18x + 7 = 3(x-3)^2 - 21 + 1 = 3(x-3)^2 - 20
    \end{equation*}
    You should check by expanding the RHS.  The approach in the textbook
    supplement is more straightforward but not as much fun.
  \end{enumerate}
\item %6 (Based on Algebra.61--66) Solve the equation.
  We try to solve the equations by factoring.
  \begin{enumerate}
  \item Factoring, we have
    \begin{equation*}
      x^2+x-6=0 \implies (x+3)(x-2)=0
    \end{equation*}
    In order for the product $(x+3)(x-2)$ to be zero, one of the factors
    must be zero, so we have $x+3=0$ which implies $x=-3$, or $x-2=0$ which
    implies $x=2$.  You should check that $x=-3$ and $x=2$ are both solutions
    to the original equation.
  \item As before, we try to solve by factoring.  
    We guess $1$, $-1$, $3$ and $-3$ as roots, but none of those work.  You
    could spend more time playing around with various attempts at factoring,
    but at some point you'll have to give up and try the quadratic formula
    instead.  In the equation $3x^2-7x+3=0$ we have $a=3$, $b=-7$, and $c=3$,
    so plugging in to the quadratic formula we get
    \begin{equation*}
      x = \frac{-b\pm \sqrt{b^2-4ac}}{2a}
      = \frac{7\pm \sqrt{49-4(3)(3)}}{2(3)}
      = \mbox{$\ds \frac{7+\sqrt{13}}{6}$ and $\ds \frac{7-\sqrt{13}}{6}$ }
    \end{equation*}
    It's a lot of work to check those answers by substituting into the
    original equation, but it would be a great exercise if you have time.
  \end{enumerate}
\item %7 (Based on Algebra.69--72) Which of the quadratics are irreducible?
  We calculate the discriminant $\Delta = b^2-4ac$:
  \begin{enumerate}
  \item 
    \begin{equation*}
      5x^2-8x+4 
      \implies \mbox{$a=5$, $b=-8$, $c=4$}
      \implies \mbox{$\Delta = 64-4(5)(4) = -16$}
    \end{equation*}
    Since the discriminant is negative, the quadratic has no roots so it
    cannot be factored, i.e., it is irreducible.
  \item 
    \begin{equation*}
      7x^2+3x-3
      \implies \mbox{$a=7$, $b=3$, $c=-3$}
      \implies \mbox{$\Delta = 9-4(7)(-3) = 93$}
    \end{equation*}
    Since the discriminant is positive, the quadratic has two (distinct) roots
    so it can be factored, i.e., it is reducible, not irreducible.
  \end{enumerate}
\item %8 (Based on Algebra.77--82) Simplify the radicals.
  \begin{enumerate}
  \item Using the rule $\sqrt{a}\sqrt{b} = \sqrt{ab}$ we have
    \begin{equation*}
      \sqrt{45}\sqrt{20} 
      = \sqrt{45\times 20} = \sqrt{3^2\times 5\times 5\times 2^2}
      = \sqrt{3^2\times 5^2 \times 2^2}
      = 3\times 5 \times 2 = 30
    \end{equation*}
  \item Using the rule $\sqrt{a}/\sqrt{b}=\sqrt{a/b}$ we have
    \begin{equation*}
      \frac{\sqrt{32x^4}}{\sqrt{2}} = \sqrt{\frac{32x^4}{2}}
      = \sqrt{16x^4} = 4x^2
    \end{equation*}
  \end{enumerate}
\item %9 (Based on Algebra.83--100) Use the Laws of Exponents to simplify.
  \begin{enumerate}
  \item Using the laws $a^m a^n = a^{m+n}$ and $a^m/a^n = a^{m-n}$ we have
    \begin{equation*}
      \frac{a^2\times a^{2n-2}}{a^{n+1}\times a^{n-1}}
      = \frac{a^{2+2n-2}}{a^{n+1+n-1}}
      = \frac{a^{2n}}{a^{2n}}
      = a^{2n-2n} = a^0 = 1
    \end{equation*}
  \item\label{prob:fracstack} %
    It may help if we re-write $x^{-1}=1/x$ and $y^{-1}=1/y$ and
    then work on the numerator and denominator separately, putting them
    over a common denominator:
    \begin{equation*}
      \frac{x^{-1}+y}{x+y^{-1}}
      = \frac{\frac{1}{x} + y}{x+\frac{1}{y}}
      = \frac{\frac{1}{x} + \frac{xy}{x}} {\frac{xy}{y}+\frac{1}{y}}
      = \frac{\frac{1+xy}{x}} {\frac{xy+1}{y}}
    \end{equation*}
    Now we invert and multiply:
    \begin{equation*}
      \frac{\frac{1+xy}{x}} {\frac{xy+1}{y}}
      = \frac{1+xy}{x} \times \frac{y}{xy+1}
      = \frac{(1+xy)y}{x(xy+1)}
      = \frac{y}{x}
    \end{equation*}
    where we have cancelled the common factor $1+xy$.  In summary,
    \begin{equation*}
      \frac{x^{-1}+y}{x+y^{-1}} = \frac{y}{x}
    \end{equation*}
    One way to partially 
    check that result might be evaluating the left and right hand
    sides for various numerical choices of $x$ and $y$.
  \end{enumerate}
\item %10 (Based on Algebra.101--106) Rationalize the denominator.
  \begin{enumerate}
  \item We multiply and divide by the ``conjugate radical'' $2+\sqrt{5}$:
    \begin{equation*}
      \frac{1}{2-\sqrt{5}} =
      \frac{1}{2-\sqrt{5}} \times \frac{2+\sqrt{5}}{2+\sqrt{5}}
      = \frac{2+\sqrt{5}}{(2-\sqrt{5})(2+\sqrt{5})}
    \end{equation*}
    The denominator is the product of a sum and a difference, so is a 
    difference of squares like 
    $(a-b)(a+b)=a^2-b^2$, and we have
    \begin{equation*}
      \frac{1}{2-\sqrt{5}} = \frac{2+\sqrt{5}}{4-5} 
      = \frac{2+\sqrt{5}}{-1} = -2-\sqrt{5}
    \end{equation*}
  \item As in the previous question, we multiply and divide by the 
    conjugate radical $\sqrt{2+h}+\sqrt{2-h}$:
    \begin{equation*}
      \frac{4}{\sqrt{2+h}-\sqrt{2-h}}
      = \frac{4}{\sqrt{2+h}-\sqrt{2-h}} \times 
      \frac{\sqrt{2+h}+\sqrt{2-h}}{\sqrt{2+h}+\sqrt{2-h}}
      = \frac{4(\sqrt{2+h}+\sqrt{2-h})}{(2+h)-(2-h)}
    \end{equation*}
    Simplifying the denominator and cancelling common factors,
    \begin{equation*}
      \frac{4}{\sqrt{2+h}-\sqrt{2-h}}
      = \frac{4(\sqrt{2+h}+\sqrt{2-h})}{2h}
      = \frac{2(\sqrt{2+h}+\sqrt{2-h})}{h}
    \end{equation*}
  \end{enumerate}
\item %11 (Based on Algebra.13--16) Expand and simplify.
  \begin{enumerate}
  \item By the binomial theorem and the distributive law,
    \begin{equation*}
      (x-2)^2 +2x(x+2)(x-4)
      = (x^2-4x+4) + 2x(x^2-4x+2x-8)
      = x^2 - 4x + 4 + 2x(x^2-2x-8)
    \end{equation*}
    Applying the distributive law again and gathering,
    \begin{equation*}
      (x-2)^2 +2x(x+2)(x-4)
      = x^2 -4x + 4 + 2x^3 -4x^2 -16x
      = 2x^3 - 3x^2 -20x + 4
    \end{equation*}
  \item We could apply the binomial theorem twice in a clever way, but
    it is more straightforward just to multiply everything out in one shot:
    \begin{equation*}
      (1-2x+x^2)^2
      = (1-2x+x^2)(1-2x+x^2)
      = 1-2x+x^2-2x+4x^2-2x^3+x^2-2x^3+x^4
    \end{equation*}
    Gathering,
    \begin{equation*}
      (1-2x+x^2)^2
      = 1-4x+6x^2-4x^3+x^4
    \end{equation*}
    Another, trickier way to solve this problem is to note that 
    $1-2x+x^2=(1-x)^2$ so 
    \begin{equation*}
      (1-2x+x^2)^2 = ((1-x)^2)^2 = (1-x)^4
      = 1-4x+6x^2-4x^3+x^4
    \end{equation*}
    by the binomial theorem.
  \end{enumerate}
\item %12 (Based on Algebra.27--28) Simplify.
  \begin{enumerate}
  \item We could answer this question like we did with 
    question~\ref{prob:fracstack}, but let's try a different method this
    time known as \textit{clearing fractions}.  We multiply the numerator
    and denominator by whatever is required to eliminate the fraction in
    the numerator, i.e.,
    \begin{equation*}
      \frac{1+\frac{1}{x-1}}{1+\frac{1}{x+1}}
      = \frac{1+\frac{1}{x-1}}{1+\frac{1}{x+1}} \times
      \frac{x-1}{x-1}
      = \frac{x-1+1}{(x-1)+\frac{x-1}{x+1}}
      = \frac{x}{(x-1)+\frac{x-1}{x+1}}
    \end{equation*}
    Similarly, we can clear the fraction in the denominator:
    \begin{equation*}
      \frac{x}{(x-1)+\frac{x-1}{x+1}}
      = \frac{x}{(x-1)+\frac{x-1}{x+1}} \times \frac{x+1}{x+1}
      = \frac{x(x+1)}{(x-1)(x+1) +(x-1)}
      = \frac{x^2+x}{x^2+x-2}
    \end{equation*}
  \item Structures like this are known as ``continued fractions''.
    We can simplify by starting at the bottom and working our way back
    up:
    \begin{equation*}
      1-\frac{1}{1-\frac{1}{1-\frac{1}{1-x}}}
      = 1-\frac{1}{1-\frac{1}{\frac{1-x}{1-x}-\frac{1}{1-x}}}
      = 1-\frac{1}{1-\frac{1}{\frac{-x}{1-x}}}
      = 1-\frac{1}{1+\frac{1-x}{x}}
    \end{equation*}
    Again, we add by rewriting $1$ as a fraction with a common denominator:
    \begin{equation*}
      1-\frac{1}{1+\frac{1-x}{x}}
      = 1 - \frac{1}{\frac{x}{x}+\frac{1-x}{x}}
      = 1 - \frac{1}{\frac{1}{x}}
      = 1-x
    \end{equation*}
  \end{enumerate}
\item %13 (Based on Algebra.42--48) Factor.
  \begin{enumerate}
  \item % EJD: Change from 8 to 64 next year
    Note that $8=2^3$ is a perfect cube, so we can use the difference
    of cubes pattern $a^3-b^3=(a-b)(a^2+ab+b^2)$ to obtain
    \begin{equation*}
      x^3-8 = (x-2)(x^2+2x+4)
    \end{equation*}
    Check by multiplying the RHS.  Now we should continue by attempting
    to factor the quadratic $x^2+2x+4$.  However, note that the discriminant is
    $\Delta = (2)^2-4(1)(4) = 4-16 = -12$ so the quadratic is irreducible
    and we can go no further.
  \item It is generally very difficult to factor cubics unless there
    is something special about the cubic.  In the previous case, the cubic
    was a difference of cubes.  In this case, we use the method we used
    in question~\ref{prob:quadAnonzero}.  We test the integer factors of the
    constant term to see whether any of them is a root of the cubic.
    The factors of $30$ are $\pm 1$, $\pm 2$, $\pm 3$, $\pm 5$, $\pm 6$,
    $\pm 10$, $\pm 15$, and $\pm 30$.  We quickly find that $-2$ is a
    root of the cubic, so we use polynomial division to extract the factor
    $x+2$:
    \begin{equation*}
      x^3 +4x^2-11x-30 
      = (x+2)(x^2+bx+c) = x^3 +bx^2+cx+2x^2+2bx+2c
      = x^3 + (b+2)x^2 + (c+2b)x + 2c
    \end{equation*}
    Comparing coefficients, we see that $b=2$ and $c=-15$, so we have
    \begin{equation*}
      x^3 +4x^2-11x-30 = (x+2)(x^2+2x-15)
    \end{equation*}
    Now we try to factor the remaining quadratic.  It is easy to guess
    factors in this case, numbers that add to $2$ and multiply to $-15$:
    \begin{equation*}
      x^3 + 4x^2 - 11x - 30 = (x+2)(x+5)(x-3)
    \end{equation*}
    It doesn't matter how you order the factors, but I prefer to order the
    factors in order of increasing root:
    \begin{equation*}
      x^3 + 4x^2 - 11x - 30 = (x+5)(x+2)(x-3)
    \end{equation*}
    Check by expanding the RHS!
  \end{enumerate}
\item %14 (Based on Algebra.52--54) Simplify the expression.
  \begin{enumerate}
  \item We must factor the numerator and the denominator.  A factor of $x$
    comes immediately out of the numerator to give 
    \begin{equation*}
      x^3 + x^2 - 6x = x(x^2+x-6) = (x+3)x(x-2)
    \end{equation*}
    (remember, I like to order factors by increasing order of the corresponding
    root).  Now to factor the denominator, we should start by trying the 
    roots of the numerator.  (If the numerator and denominator don't have a
    root in common, they don't have a factor in common, and we can't simplify
    so we could just stop.)  Immediately we find that $x=2$ is a root of 
    the denominator, and extracting that root by polynomial division gives
    \begin{equation*}
      3x^2-8x+4 = (x-2)(ax+b) = ax^2 + (b-2a)x - 2b
    \end{equation*}
    which tells us that $a=3$ and $b=-2$.  You should check that
    \begin{equation*}
      3x^2-8x+4 = (3x-2)(x-2)
    \end{equation*}
    In summary,
    \begin{equation*}
      \frac{x^3+x^2-6x}{3x^2-8x+4}
      = \frac{(x+3)x(x-2)}{(3x-2)(x-2)}
      = \frac{(x+3)x}{3x-2}
    \end{equation*}
    You could expand the numerator, but there's no particular reason to
    do so, so I would just leave it alone.
  \item % EJD: change this so x-1 turns out to be factor of numerator: try 4
    We often find that we can make simplifications if we can factor
    the polynomials involved in expressions, particularly polynomials in the
    denominator.  Note that $x^2-6x+5 = (x-1)(x-5)$ so we have
    \begin{equation*}
      \frac{x}{x-1} + \frac{1}{x^2-6x+5}
      = \frac{x}{x-1} + \frac{1}{(x-1)(x-5)}
      = \frac{x(x-5)}{(x-1)(x-5)} + \frac{1}{(x-1)(x-5)}
      = \frac{x^2-5x+1}{(x-1)(x-5)}
    \end{equation*}
    Neither $x=1$ nor $x=5$ is a root of the numerator, so we cannot simplify
    any further.  You could try factoring the numerator, but in my opinion
    that's unnecessarily messy.
  \end{enumerate}
\item %15 (Based on Algebra.67--68) Solve the equation.
  \begin{enumerate}
  \item As usual with cubics, we should try the factors of the constant term
    as roots.  The factors of $2$ are $\pm 1$, $\pm 2$, and we find that
    $x=1$ is a root, so we get
    \begin{equation*}
      x^3-3x+2 = (x-1)(x^2+bx+c) = x^3 + bx^2 + cx -x^2 -bx - c 
      = x^3 +(b-1)x^2 + (c-b)x - c
    \end{equation*}
    Comparing coefficients gives $b=1$ and $c=-2$; factoring the resulting
    quadratic gives
    \begin{equation*}
      x^3-3x+2 = (x-1)(x^2+x-2) = (x-1)(x+2)(x-1) = (x+2)(x-1)^2
    \end{equation*}
    (Check by expanding the RHS.)  So the equation can be written
    \begin{equation*}
      x^3-3x+2 = 0 \implies (x+2)(x-1)^2 = 0
    \end{equation*}
    from which we see that the solutions to the equation are $x=-2$ and
    $x=1$ (which is a ``double root'').
  \item % EJD: Change x to -x so root is x=-1
    Again, we try the factors of the constant terms as roots of the
    cubic.  The factors are $\pm 1$, $\pm 3$, and we soon see that 
    $x=1$ works, so we have
    \begin{equation*}
      x^3-6x^2+8x-3 = (x-1)(x^2+bx+c) = x^3 +bx^2+cx - x^2 -bx -c
      = x^3 + (b-1) x^2 + (c-b) x -c 
    \end{equation*}
    Comparing coefficients gives $b=-5$, $c=3$ so
    \begin{equation*}
      x^3 - 6x^2 + 8x -3 = 0 \implies (x-1)(x^2-5x+3) = 0
    \end{equation*}
    Unlike the previous problem, in this case we can't factor the remaining
    quadratic, so we need to use the quadratic formula to get an expression
    for the roots.  In summary, the solutions to $x^3-6x^2+8x-3$ are
    \begin{equation*}
      \mbox{$\ds x = 1$, $\ds \frac{5-\sqrt{13}}{2}$, 
        $\ds \frac{5+\sqrt{13}}{2}$}
    \end{equation*}
  \end{enumerate}
\item %16 (Based on Algebra.73--76) Use the binomial theorem to expand 
  % the expression.
  \begin{enumerate}
  \item Plugging $n=5$ into the binomial theorem we get
    \begin{equation*}
      (x+y)^5 = x^5 + 5x^4y + 10x^3y^2 + 10x^2y^3 + 5xy^4 + y^5
    \end{equation*}
  \item By the binomial theorem we have
    \begin{equation*}
      (x^2-1)^3 = (x^2)^3 + 3(x^2)^2(-1) + 3(x^2)(-1)^2 + (-1)^3
    \end{equation*}
    By the laws of exponents we have
    \begin{equation*}
      (x^2-1)^3 = x^6 - 3x^4 + 3x^2 -1
    \end{equation*}
  \end{enumerate}
% EJD: reorder these problems: factoring cubics is much harder than
% binomial theorem or conjugate radicals
\item %17 (Based on Algebra.107--108) Rationalize the expression.
  \begin{enumerate}
  \item Multiplying and dividing by the conjugate radical gives
    \begin{equation*}
      \sqrt{x^2+2}-x = (\sqrt{x^2+2}-x) \times 
        \frac{ \sqrt{x^2+2}+x }{ \sqrt{x^2+2}+x }
      = \frac{x^2+2 - x^2}{ \sqrt{x^2+2}+x }
      = \frac{2}{ \sqrt{x^2+2}+x }
    \end{equation*}
  \item Again, multiplying and dividing by the conjugate radical gives
    \begin{equation*}
      \sqrt{2x^2+3x}-\sqrt{2x^2-3x}
      = (\sqrt{2x^2+3x}-\sqrt{2x^2-3x}) \times
        \frac{ \sqrt{2x^2+3x}+\sqrt{2x^2-3x} }{ \sqrt{2x^2+3x}+\sqrt{2x^2-3x} }
    \end{equation*}
    Multiplying the numerators,
    \begin{equation*}
      \sqrt{2x^2+3x}-\sqrt{2x^2-3x}
      = \frac{ (2x^2+3x)-(2x^2-3x) }{ \sqrt{2x^2+3x}+\sqrt{2x^2-3x} }
      = \frac{ 6x }{ \sqrt{2x^2+3x}+\sqrt{2x^2-3x} }
    \end{equation*}
  \end{enumerate}
\end{enumerate}
\end{document}

