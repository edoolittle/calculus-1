\documentclass[12pt]{article}
\usepackage{mathptmx}
\usepackage{fullpage}
\usepackage{multicol}
\usepackage{amsmath,amssymb}
\usepackage{graphicx}
\usepackage[aux]{rerunfilecheck}

\newcommand{\ds}{\displaystyle}

\reversemarginpar

\title{MATH110-S01-S02 200930 Quiz 0 Solutions}
\author{Edward Doolittle}

\begin{document}
\maketitle

\begin{enumerate}
\item We can work this out in stages.  We have
  \begin{align*}
    f(a) &= 2a^2-3 \\
    f(a+h) &= 2(a+h)^2-3 = 2(a^2+2ah+h^2) - 3 = 2a^2 + 4ah + 2h^2-3 \\
    f(a+h) - f(a) &= 2a^2+4ah + 2h^2- 3 - (2a^2-3) = 2a^2+4ah + 2h^2 - 3
    - 2a^2+3 = 4ah + 2h^2
  \end{align*}
  so 
  \begin{equation*}
    \frac{f(a+h)-f(a)}{a+h-a} 
    = \frac{4ah+2h^2}{h}
  \end{equation*}
  You can cancel a common factor of $h$ from \textit{every} term in the
  above expression to obtain a final answer of $4a+2h$, 
  but that cancellation isn't really one hundred percent 
  correct unless you know that $h\ne 0$.
\item 
  \begin{enumerate}
  \item We write $g(F)$ in standard form by expanding the bracket:
    \begin{equation*}
      g(F) = \frac{5}{9} F - \frac{160}{9}
    \end{equation*}
    The slope is the coefficient of $F$, which is $5/9$.  It represents
    the change in the Celsius temperature for each one degree change in
    the Fahrenheit temperature.
  \item Based on the above standard form, the $C$ intercept is $-160/9
    \approx -17.78$.  That represents the temperature in Celsius corresponding
    to the temperature of $0$ degrees Fahrenheit.
  \item We have
    \begin{equation*}
      h\circ g(F) = h(g(F)) = \frac{9}{5} (g(F)) + 32
      = \frac{9}{5} \left(\frac{5}{9} (F-32) \right) + 32
      = F-32 + 32 = F
    \end{equation*}
    We can say that $h\circ g$ is the identity function.  That makes sense
    because $g$ takes a temperature in Fahrenheit and converts it to 
    Celsius, while $h$ converts the Celsius temperature back to Fahrenheit,
    so the composition should bring us back where we started.
  \end{enumerate}
\end{enumerate}

\end{document}


