\documentclass[12pt]{article}
\usepackage{mathptmx}
\usepackage{fullpage}
\usepackage{multicol}
\usepackage{amsmath,amssymb}
\usepackage[aux]{rerunfilecheck}

\newcommand{\ds}{\displaystyle}

\reversemarginpar

\usepackage{lastpage,fancyhdr}
\usepackage{fancyhdr}
\pagestyle{fancy}
\lhead{MATH110-S01-S02 200930 Quiz 0      \\
  Time: 20 minutes                        \\ \quad }
\chead{Page\ \thepage\ of \pageref{LastPage}   \\ \quad \\ \quad}
\rhead{Name: \underline{\hspace{1.5in}}        \\
  Student \#: \underline{\hspace{1.5in}}  \\ \quad }
\cfoot{}
\addtolength{\headheight}{\baselineskip}
\addtolength{\headheight}{\baselineskip}
\addtolength{\headheight}{\baselineskip}
\addtolength{\headheight}{\baselineskip}
\renewcommand{\headrulewidth}{0pt}
\fancypagestyle{plain}{%
  \lhead{}
  \chead{FIRST NATIONS UNIVERSITY                \\
    DEPARTMENT OF SCIENCE \\
    MATH110-S01-S02 200930 Quiz 0   \\
    \quad                                      }
  \rhead{}
  \cfoot{Page\ \thepage\ of \pageref{LastPage}}
}

\begin{document}
\thispagestyle{plain}

\begin{flushleft}
Time:  20 minutes                \hfill       Name: \underline{\hspace{2in}} \\
Instructor: Dr. Edward Doolittle \hfill Student \#: \underline{\hspace{2in}}
\end{flushleft}

\noindent
Please\marginpar{\centering (marks)} do questions 1 and 2.  You have 10 minutes
to do each question, for a total of 20
minutes for the quiz.  A 
calculator of the type mentioned in the course outline is allowed; no other
aids allowed.
%but is not necessary.  
%You may leave early if you can
%do so without disturbing any of your colleagues.
%If you finish early, I suggest you check your work thoroughly.
%\textbf{Please do not disturb your colleagues by climbing over them while
%they are trying to write the quiz.}

\begin{enumerate}
\item For\marginpar{\centering (10)} 
  the function $f(x)=2x^2-3$, evaluate the difference quotient
  \begin{equation*}
    \frac{f(a+h)-f(a)}{a+h-a}
  \end{equation*}
\vfill
\newpage
\item The\marginpar{\centering (10)} 
  relationship between the Celsius ($C$) and Fahrenheit ($F$) temperature
  scales is given by the linear function $C=g(F)$ where 
  \begin{equation*}
    g(F) = \frac{5}{9}(F-32)
  \end{equation*}
  \begin{enumerate}
  \item What is the slope of the graph of $g$ and what does it represent?
  \item what is the $C$-intercept of the graph of $g$ and what does it 
    represent?
  \item Consider the function $F=h(C)$ where
    \begin{equation*}
      h(C) = \frac{9}{5}C + 32
    \end{equation*}
    Find the composition $h\circ g$.
  \end{enumerate}
\vfill
\end{enumerate}

\end{document}

