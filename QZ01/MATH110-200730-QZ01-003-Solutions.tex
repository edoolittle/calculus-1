\documentclass{article}
\usepackage{fullpage}
\usepackage{graphicx}
\usepackage[aux]{rerunfilecheck}

\newcommand{\ds}{\displaystyle}

\title{MATH 110 Quiz 1 Section 003 Solutions}
\author{Edward Doolittle}

\begin{document}
\maketitle
\begin{enumerate}
\item
  \begin{enumerate}
  \item The table is as follows.  The $x-16$ column is of course trivial,
    as is the $f(x)$ column once the other two are done (just shifting
    decimal points).
    \begin{center}
      \begin{tabular}{|l|l|l|l|}
        \hline
	\rule{10pt}{0pt}$x$\rule{10pt}{0pt}     
	  & \rule{10pt}{0pt}\rule{0pt}{12pt}$4-\sqrt{x}$\rule{10pt}{0pt}
	  & \rule{10pt}{0pt}$x-16$\rule{10pt}{0pt}
	  & \rule{20pt}{0pt}$f(x)$\rule{20pt}{0pt} \\
	\hline
	\rule{0pt}{12pt}$17.00$ &   -0.1231    & 1.0000 &  -0.1231   \\
	\hline
	\rule{0pt}{12pt}$16.10$ &   -0.01248   & 0.1000 &  -0.1248   \\
	\hline
	\rule{0pt}{12pt}$16.01$ &   -0.001250  & 0.0100 &  -0.1250   \\
	\hline
	\rule{0pt}{12pt}$15.99$ &    0.001250  &-0.0100 &  -0.1250   \\
	\hline
	\rule{0pt}{12pt}$15.90$ &    0.01252   &-0.1000 &  -0.1252   \\
	\hline
	\rule{0pt}{12pt}$15.00$ &    0.1270    &-1.0000 &  -0.1270   \\
	\hline
      \end{tabular}
    \end{center}
  \item % 1(b)
    Based on the above table, a reasonable guess for the limit would be
    $-0.1250$ to four decimal places.  (Using limit theorems, we can now
    show that that is the exact answer.)
  \end{enumerate}
\item % 2
  Candidates for vertical asymptotes of rational functions 
  are vertical lines over $x$ values at which the denominator goes to $0$.
  Factoring the denominator, we have $\ds y = \frac{x+1}{(x+3)(x-2)}$ so
  our candidates for asymptotes are the lines $x=-3$ and $x=2$.  The
  rational function may have a removable discontinuity at those $x$ values
  rather than an infinite discontinuity, however, so we must check (one-sided)
  limits as $x$ approaches those candidate values.

  For $\ds \lim_{x\to -3^-}$ we have $x+1$ and $x-2$ both negative and non-zero,
  and $x+3$ negative and close to zero.  Three negatives make a negative,
  and a number in the denominator close to zero gives a large overall
  result, so $\ds \lim_{x\to -3^-} \frac{x+1}{(x+3)(x-2)} = -\infty$.

  For $\ds \lim_{x\to -3^+}$, $x+1$ and $x-2$ are again both negative and
  non-zero, but $x+3$ is positive and close to zero, giving the limit
  $\ds \lim_{x\to -3^+} \frac{x+1}{(x+3)(x-2)} = +\infty$.

  For $\ds \lim_{x\to 2^-}$, $x+1$ and $x+3$ are both positive, and
  $x-2$ is negative and close to zero, so overall we have a limit
  of $-\infty$.  For $\ds \lim_{x\to 2^+}$ we have a limit of $+\infty$.

  In summary, there are vertical asymptotes at $x=-3$ and $x=2$.
\end{enumerate}

\end{document}


\begin{enumerate}
\item 
  \begin{enumerate}
  \item Let\marginpar{\centering (10)}
    $\ds f(x)=\frac{4-\sqrt{x}}{x-16}$.
    Use your calculator to 
    fill in the following table of values to 4 decimal points.
    \begin{center}
      \begin{tabular}{|l|l|l|l|}
        \hline
	\rule{10pt}{0pt}$x$\rule{10pt}{0pt}     
	  & \rule{10pt}{0pt}\rule{0pt}{12pt}$4-\sqrt{x}$\rule{10pt}{0pt}
	  & \rule{10pt}{0pt}$x-16$\rule{10pt}{0pt}
	  & \rule{20pt}{0pt}$f(x)$\rule{20pt}{0pt} \\
	\hline
	\rule{0pt}{12pt}$17.00$ &              &        &        \\
	\hline
	\rule{0pt}{12pt}$16.10$ &              &        &        \\
	\hline
	\rule{0pt}{12pt}$16.01$ &              &        &        \\
	\hline
	\rule{0pt}{12pt}$15.99$ &              &        &        \\
	\hline
	\rule{0pt}{12pt}$15.90$ &              &        &        \\
	\hline
	\rule{0pt}{12pt}$15.00$ &              &        &        \\
	\hline
      \end{tabular}
    \end{center}
  \item Use the table of values to estimate the limit
    $\displaystyle\lim_{x\to 16} \frac{4-\sqrt{x}}{x-16}$.
  \end{enumerate}
\newpage
\item Find\marginpar{\centering (10)} the vertical asymptotes of the function
  $\displaystyle y = \frac{x+1}{x^2+x-6}$.
\end{enumerate}

\end{document}

