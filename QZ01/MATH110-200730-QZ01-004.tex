\documentclass[12pt]{article}
\usepackage{mathptmx}
\usepackage{fullpage}
\usepackage{multicol}
\usepackage{amsmath,amssymb}
\usepackage[aux]{rerunfilecheck}

\newcommand{\ds}{\displaystyle}

\reversemarginpar

\usepackage{lastpage,fancyhdr}
\usepackage{fancyhdr}
\pagestyle{fancy}
\lhead{MATH 110 200710 Quiz 1 (004)     \\
  Time: 20 minutes                        \\ \quad }
\chead{Page\ \thepage\ of \pageref{LastPage}   \\ \quad \\ \quad}
\rhead{Name: \underline{\hspace{1.5in}}        \\
  Student \#: \underline{\hspace{1.5in}}  \\ \quad }
\cfoot{}
\addtolength{\headheight}{\baselineskip}
\addtolength{\headheight}{\baselineskip}
\addtolength{\headheight}{\baselineskip}
\addtolength{\headheight}{\baselineskip}
\renewcommand{\headrulewidth}{0pt}
\fancypagestyle{plain}{%
  \lhead{}
  \chead{UNIVERSITY OF REGINA                \\
    DEPARTMENT OF MATHEMATICS AND STATISTICS \\
    MATH 110 200730 Quiz 1 (Section 004)     \\
    \quad                                      }
  \rhead{}
  \cfoot{Page\ \thepage\ of \pageref{LastPage}}
}

\begin{document}
\thispagestyle{plain}

\begin{flushleft}
Time:  20 minutes                \hfill       Name: \underline{\hspace{2in}} \\
Instructor: Dr. Edward Doolittle \hfill Student \#: \underline{\hspace{2in}}
\end{flushleft}

\noindent
Please\marginpar{\centering (marks)} do questions 1 and 2.  You have 10 minutes
to do each question, for a total of 20
minutes for the quiz.  A non-programmable
calculator of the type mentioned in the course outline is allowed.
%but is not necessary.  
%You may leave early if you can
%do so without disturbing any of your colleagues.
If you finish early, I suggest you check your work thoroughly.
\textbf{Please do not disturb your colleagues by climbing over them while
they are trying to write the quiz.}

\begin{enumerate}
\item 
  \begin{enumerate}
  \item Let\marginpar{\centering (10)}
    $\ds f(x)=\frac{4-\sqrt{x}}{x-16}$.
    Use your calculator to 
    fill in the following table of values to 4 decimal points.
    \begin{center}
      \begin{tabular}{|l|l|l|l|}
        \hline
	\rule{10pt}{0pt}$x$\rule{10pt}{0pt}     
	  & \rule{10pt}{0pt}\rule{0pt}{12pt}$4-\sqrt{x}$\rule{10pt}{0pt}
	  & \rule{10pt}{0pt}$x-16$\rule{10pt}{0pt}
	  & \rule{20pt}{0pt}$f(x)$\rule{20pt}{0pt} \\
	\hline
	\rule{0pt}{12pt}$17.00$ &              &        &        \\
	\hline
	\rule{0pt}{12pt}$16.10$ &              &        &        \\
	\hline
	\rule{0pt}{12pt}$16.01$ &              &        &        \\
	\hline
	\rule{0pt}{12pt}$15.99$ &              &        &        \\
	\hline
	\rule{0pt}{12pt}$15.90$ &              &        &        \\
	\hline
	\rule{0pt}{12pt}$15.00$ &              &        &        \\
	\hline
      \end{tabular}
    \end{center}
  \item Use the table of values to estimate the limit
    $\displaystyle\lim_{x\to 16} \frac{4-\sqrt{x}}{x-16}$.
  \end{enumerate}
\newpage
\item Find\marginpar{\centering (10)} the vertical asymptotes of the function
  $\displaystyle y = \frac{x+1}{x^2+x-6}$.
\end{enumerate}

\end{document}

