\documentclass[12pt]{article}
\usepackage{mathptmx}
\usepackage{fullpage}
\usepackage{multicol}
\usepackage{amsmath,amssymb}
\usepackage[aux]{rerunfilecheck}

\newcommand{\ds}{\displaystyle}

\reversemarginpar

\title{MATH 110-003 200730 Quiz 2 Solutions}
\author{Edward Doolittle}

\begin{document}
\maketitle

\begin{enumerate}
\item The function under the limit can be rewritten as
  \begin{displaymath}
    \frac{(5+h)^{-1}-5^{-1}}{h}
    = \frac{1}{h} \left(\frac{1}{5+h}-\frac{1}{5}\right)
    = \frac{1}{h} \frac{5-(5+h)}{(5+h)(5)}
    = \frac{1}{h} \frac{-h}{5(5+h)}
  \end{displaymath}
  (5 marks).  Therefore the limit can be evaluated as follows:
  \begin{displaymath}
    \lim_{h\to 0} \frac{(5+h)^{-1}-5^{-1}}{h}
    = \lim_{h\to 0} \frac{1}{h} \frac{-h}{5(5+h)} 
    = \lim_{h\to 0} \frac{-1}{5(5+h)}
    = -\frac{1}{25}
  \end{displaymath}
  (5 marks)
\item The $x$ values of interest are the points where the denominator
  of $\ds \frac{x-3}{2x^2-5x-3}$ is zero, and the points where the definition
  of the function changes.  At all other points the function is guaranteed
  to be continuous by the theorem on the continuity of rational functions.

  The denominator can be factored as $2x^2-5x-3=(2x+1)(x-3)$, so the values
  of interest are $x=-1/2$ and $x=3$ (the latter for two reasons).  To
  determine whether $f$ is continuous at those values we have to take limits
  and. if necessary, compare with the values of the function.
  (2 marks)

  For $x=-1/2$, we have
  \begin{displaymath}
    \lim_{x\to -1/2^-} f(x)
    = \lim_{x\to -1/2^-} \frac{x-3}{(2x+1)(x-3)}
    = \lim_{x\to -1/2^-} \frac{1}{2x+1}
    = -\infty
  \end{displaymath}
  since the denominator is a negative number close to zero.  Therefore
  $\ds\lim_{x\to -1/2} f(x)$ does not exist and $f$ is not continuous
  at $x=-1/2$.  The type of discontinuity is infinite.  (We don't have
  to check the limit $x\to -1/2^+$, but you can if you want; you should
  get $+\infty$.) (4 marks)

  For $x=3$ we have 
  \begin{displaymath}
    \lim_{x\to 3} f(x)
    = \lim_{x\to 3} \frac{x-3}{(2x+1)(x-3)}
    = \lim_{x\to 3} \frac{1}{2x+1}
    = \frac{1}{7}
  \end{displaymath}
  and $f(3) = 1/7$ by the definition of the function.  Since the value
  of the function agrees with the limit at $3$, $f$ is continuous at $3$.
  (4 marks)

  In summary, the only discontinuity of $f$ is at $-1/2$, and the type of
  that discontinuity is infinite.
\end{enumerate}

\end{document}


