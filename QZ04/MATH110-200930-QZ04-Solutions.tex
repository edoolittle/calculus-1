\documentclass[12pt]{article}
\usepackage{mathptmx}
\usepackage{fullpage}
\usepackage{multicol}
\usepackage{amsmath,amssymb}
\usepackage[aux]{rerunfilecheck}

\newcommand{\ds}{\displaystyle}

\reversemarginpar

\title{MATH110--S01--S02 200930 Quiz 4 Solutions}
\author{Edward Doolittle}

\begin{document}
\maketitle

\begin{enumerate}
\item By the quotient rule,
  \begin{equation*}
    f'(x) = \frac{\left(\frac{d}{dx} \cos x\right) (\tan x + \sin x)
    - \cos x \left( \frac{d}{dx} (\tan x + \sin x)\right)}{(\tan x + \sin x)^2}
  \end{equation*}
  By the derivative of $\cos$, the sum rule, and the derivatives of $\tan$
  and $\sin$, we have
  \begin{equation*}
    f'(x) = \frac{-\sin x(\tan x + \sin x) - \cos x (\sec^2 x + \cos x)}{
      (\tan x + \sin x)^2}
  \end{equation*}
  Various simplifications are possible here, but are not necessary.  For
  example, multiplying the numerator out and applying the Pythagorean identity,
  \begin{equation*}
    f'(x) = \frac{-\sin x \tan x - \sin^2 x - \sec x - \cos^2 x}{
      (\tan x + \sin x )^2}
    = \frac{-\sin x \tan x -\sec x -1}{(\tan x + \sin x)^2}
  \end{equation*}
  Further simplification might be possible by multiplying the numerator and
  denominator through by $\cos^2 x$, etc.
\item Call the limit in the question $L$.
  By the chain rule for limits (or the product rule for limits), we have
  \begin{equation*}
    L = \lim_{t\to 0} \left( \frac{\sin 3t}{t} \right)^2
    = \left( \lim_{t\to 0} \frac{\sin 3t}{t} \right)^2
  \end{equation*}
  provided the limit inside the bracket exists.  But it is just a variation
  on the basic trig limit we studied:
  \begin{equation*}
    \lim_{t\to 0} \frac{\sin 3t}{t}
    = \lim_{t\to 0} 3 \frac{\sin 3t}{3t}
    = 3 \lim_{t\to 0} \frac{\sin 3t}{3t}
    = 3 \cdot 1
  \end{equation*}
  Putting it all together, $L=9$.
\end{enumerate}

\end{document}


