\documentclass[12pt]{article}
\usepackage{mathptmx}
\usepackage{fullpage}
\usepackage{multicol}
\usepackage{amsmath,amssymb}
\usepackage[aux]{rerunfilecheck}

\newcommand{\ds}{\displaystyle}

\reversemarginpar

\usepackage{lastpage,fancyhdr}
\usepackage{fancyhdr}
\pagestyle{fancy}
\lhead{MATH 110 200710 Quiz 5 (003)     \\
  Time: 20 minutes                        \\ \quad }
\chead{Page\ \thepage\ of \pageref{LastPage}   \\ \quad \\ \quad}
\rhead{Name: \underline{\hspace{1.5in}}        \\
  Student \#: \underline{\hspace{1.5in}}  \\ \quad }
\cfoot{}
\addtolength{\headheight}{\baselineskip}
\addtolength{\headheight}{\baselineskip}
\addtolength{\headheight}{\baselineskip}
\addtolength{\headheight}{\baselineskip}
\renewcommand{\headrulewidth}{0pt}
\fancypagestyle{plain}{%
  \lhead{}
  \chead{UNIVERSITY OF REGINA                \\
    DEPARTMENT OF MATHEMATICS AND STATISTICS \\
    MATH 110 200730 Quiz 5 (Section 003)     \\
    \quad                                      }
  \rhead{}
  \cfoot{Page\ \thepage\ of \pageref{LastPage}}
}

\begin{document}
\thispagestyle{plain}

\begin{flushleft}
Time:  20 minutes                \hfill       Name: \underline{\hspace{2in}} \\
Instructor: Dr. Edward Doolittle \hfill Student \#: \underline{\hspace{2in}}
\end{flushleft}

\noindent
Please\marginpar{\centering (marks)} do questions 1 and 2.  You have 10 minutes
to do each question, for a total of 20
minutes for the quiz.  A non-programmable
calculator of the type mentioned in the course outline is allowed.
%but is not necessary.  
%You may leave early if you can
%do so without disturbing any of your colleagues.
If you finish early, I suggest you check your work thoroughly.
\textbf{Please do not disturb your colleagues by climbing over them while
they are trying to write the quiz.}

\begin{enumerate}
\item At\marginpar{\centering (10)} 
  noon, ship $A$ is $35$ km west of ship $B$.  Ship $A$ is sailing
  west at $10$ km/h and ship $B$ is sailing north at $25$ km/h.  How fast
  is the distance between the ships changing at 4:00 pm that day?
\newpage
\item The\marginpar{\centering (10)}
  \textbf{circumference} of a sphere was measured to be $84$ cm with
  a possible error of $0.5$ cm.
  \begin{enumerate}
  \item Use differentials to estimate the maximum error in the calculated
    surface area.  What is the relative error?
  \item Use differentials to estimate the maximum error in the calculated
    volume.  What is the relative error?
  \end{enumerate}
  Note: in relation to the radius of a sphere, 
  the circumference is $C=2\pi r$, the surface area is $A=4\pi r^2$, and
  the volume is $\ds V=(4/3) \pi r^3$.
\end{enumerate}

\end{document}

