\documentclass[12pt]{article}
\usepackage{mathptmx}
\usepackage{fullpage}
\usepackage{multicol}
\usepackage{amsmath,amssymb}
\usepackage[aux]{rerunfilecheck}

\newcommand{\ds}{\displaystyle}

\reversemarginpar

\usepackage{lastpage,fancyhdr}
\usepackage{fancyhdr}
\pagestyle{fancy}
\lhead{MATH 110 200710 Quiz 7 (003)     \\
  Time: 20 minutes                        \\ \quad }
\chead{Page\ \thepage\ of \pageref{LastPage}   \\ \quad \\ \quad}
\rhead{Name: \underline{\hspace{1.5in}}        \\
  Student \#: \underline{\hspace{1.5in}}  \\ \quad }
\cfoot{}
\addtolength{\headheight}{\baselineskip}
\addtolength{\headheight}{\baselineskip}
\addtolength{\headheight}{\baselineskip}
\addtolength{\headheight}{\baselineskip}
\renewcommand{\headrulewidth}{0pt}
\fancypagestyle{plain}{%
  \lhead{}
  \chead{UNIVERSITY OF REGINA                \\
    DEPARTMENT OF MATHEMATICS AND STATISTICS \\
    MATH 110 200730 Quiz 7 (Section 003)     \\
    \quad                                      }
  \rhead{}
  \cfoot{Page\ \thepage\ of \pageref{LastPage}}
}

\begin{document}
\thispagestyle{plain}

\begin{flushleft}
Time:  20 minutes                \hfill       Name: \underline{\hspace{2in}} \\
Instructor: Dr. Edward Doolittle \hfill Student \#: \underline{\hspace{2in}}
\end{flushleft}

\noindent
Please\marginpar{\centering (marks)} do questions 1 and 2.  You have 10 minutes
to do each question, for a total of 20
minutes for the quiz.  A non-programmable
calculator of the type mentioned in the course outline is allowed.
%but is not necessary.  
%You may leave early if you can
%do so without disturbing any of your colleagues.
If you finish early, I suggest you check your work thoroughly.
\textbf{Please do not disturb your colleagues by climbing over them while
they are trying to write the quiz.}

\begin{enumerate}
\item Find\marginpar{\centering (10)} 
  $\ds f(x)$ given that $\ds f''(x)=8-12x+60x^3$, $\ds f(0)=-1$, $\ds f'(0)=5$.
\vfill
\newpage
\item 
  \begin{enumerate}
  \item Apply\marginpar{\centering (10)} Newton's method to the 
    equation $\ds \frac{1}{x}-9=0$ to derive the following algorithm
    for calculating $\ds \frac{1}{9}$.
    \begin{align*}
      x_{n+1} = 2x_n-9x_n^2
    \end{align*}
\vfill
\vfill
\vfill
\vfill
  \item Starting with the initial approximation $\ds x_1=0.1$, find $\ds x_2$.
\vfill
\vfill
\vfill
  \item Bonus: what is the pattern for $\ds x_n$ in the above case?
\vfill
  \end{enumerate}
\end{enumerate}

\end{document}

