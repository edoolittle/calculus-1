\documentclass[12pt]{article}
\usepackage{mathptmx}
\usepackage{fullpage}
\usepackage{multicol}
\usepackage{amsmath,amssymb}
\usepackage{graphicx}
\usepackage[aux]{rerunfilecheck}

\newcommand{\ds}{\displaystyle}

\reversemarginpar

\title{MATH 110-004 200730 Quiz 7 Solutions}
\author{Edward Doolittle}

\begin{document}
\maketitle

\begin{enumerate}
\item 
  \textbf{Method 1:}  Finding antiderivatives,
  $f'(x)=C+8x-6x^2+15x^4$ and $f(x)=D+Cx+4x^2-2x^3+3x^5$.  (Check
  by differentiating!)  Then we have $f'(0)=C$ on the one hand,
  and $f'(0)=5$ by the given data, so $C=5$.  Similarly, we have
  $f(0)=D$ on the one hand, and $f(0)=-1$ by the given data, so
  $f(x)=-1+5x+4x^2-2x^3+3x^5$.
  
  \textbf{Method 2:}  In more complicated problems, it is probably
  best to try to figure out as much as possible about the constants
  after each antidifferentiation.  In this case, it doesn't matter,
  and in some cases it doesn't help, but let's try it this way just
  for fun.  We have $f'(x)=C+8x-6x^2+15x^4$ as above, but \textit{now}
  we use the information $f'(0)=5$ to obtain $f'(x)=5+8x-6x^2+15x^4$.
  Finding the antiderivative of that, $f(x)=D+5x+4x^2-2x^3+3x^5$.
  Now we use $f(0)=-1$ to obtain $D=-1$, so the final answer is
  $f(x)=-1+5x+4x^2-2x^3+3x^5$.
\item 
  \begin{enumerate}
  \item (8 marks) Let $\ds f(x)=\frac{1}{x}-9$.  
    Then $\ds f'(x)=-\frac{1}{x^2}$.
    The formula for Newton's method is
    \begin{align*}
      x_{n+1} &= x_n - \frac{f(x_n)}{f'(x_n)}
      = x_n - \frac{\ds \frac{1}{x_n}-9}{\ds -\frac{1}{x_n^2}}
      = x_n + \frac{\ds \frac{x_n^2}{x_n}-9x_n^2}{\ds \frac{x_n^2}{x_n^2}}
      \\
      &= x_n + x_n - 9x_n^2 = 2x_n-9x_n^2.
    \end{align*}
  \item (2 marks) With $x_1=0.1$ we have
    \begin{align*}
      x_2 = 2(0.1) - 9(0.1)^2 = 0.2-0.09 = 0.11.
    \end{align*}
  \item (1 mark) The number of digits of accuracy doubles, roughly,
    with each iteration of Newton's method, 
    so a reasonable guess is that $x_n$ is
    $2^{n-1}$ $1$s after the decimal point.  A quick check of the first
    few iterations should reinforce that guess.  (It is possible to prove
    that that guess is correct using the expression 
    \begin{align*}
      x_n=\frac{\ds 10^{2^{n-1}}-1}{\ds 9\cdot 10^{2^{n-1}}}
    \end{align*}
    and mathematical induction, but that is not necessary to get the 
    bonus mark.)
  \end{enumerate}
\end{enumerate}

\end{document}


