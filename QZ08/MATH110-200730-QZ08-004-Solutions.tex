\documentclass[12pt]{article}
\usepackage{mathptmx}
\usepackage{fullpage}
\usepackage{multicol}
\usepackage{amsmath,amssymb}
\usepackage{graphicx}
\usepackage[aux]{rerunfilecheck}

\newcommand{\ds}{\displaystyle}

\reversemarginpar

\title{MATH 110-004 200730 Quiz 8 Solutions}
\author{Edward Doolittle}

\begin{document}
\maketitle

\begin{enumerate}
\item The translation is straightforward: in the limit $\Delta x$ becomes $dx$;
  the rest of the summand, $2-5(x_i^*)^2+\cos(x_i^*)$,
  becomes the integrand $2-5x^2+\cos x$; the summation symbol becomes the
  integration sign; and the bounds of integration are supplied by the interval
  over which you integrate.  So an answer is
  \begin{align*}
    \int_0^\pi (2-5x^2+\cos x) \; dx
  \end{align*}
  You can check the answer by forming a Riemann sum for the above definite
  integral and comparing with the question.
\item The definition of a definite integral says
  \begin{align*}
    \int_0^1 (x^2+3) \; dx
    = \lim_{n\to\infty} \sum_{i=1}^n \left( (x_i^*)^2 + 3 \right) \Delta x
  \end{align*}
  If we divide the interval $[0,1]$ into $n$ equal subdivisions we have
  $\Delta x=(1-0)/n=1/n$ and enpoints of the intervals given by
  $x_i=0+(i/n)(1-0) = i/n$.  If we choose as sample points the right-hand
  endpoints of the intervals we have $x_i^* = x_i$.  (Other choices are
  possible, but unless another choice is specifically requested, the 
  right-hand endpoints are easiest to work with.)  Putting all of that 
  information in to the above formula, we have
  \begin{align*}
    \int_0^1 (x^2+3) \; dx
    = \lim_{n\to\infty}\sum_{i=1}^n\left(\left(\frac{i}{n}\right)^2+3\right)
    \frac{1}{n}
    = \lim_{n\to\infty}\sum_{i=1}^n \left(\frac{i^2}{n^3} + \frac{3}{n}\right)
  \end{align*}
  Using rules for evaluating summations, and the supplied formulas, we have
  \begin{align*}
    \int_0^1 (x^2+3) \; dx
    = \lim_{n\to\infty} \left( \frac{1}{n^3} \sum_{i=1}^n i^2
    + \frac{1}{n} \sum_{i=1}^n 3 \right)
    = \lim_{n\to\infty} \left( \frac{1}{n^3} \frac{n(n+1)(2n+1)}{6}
    + \frac{1}{n} 3n \right)
    = \frac{2}{6} + 3 = \frac{10}{3}
  \end{align*}
\end{enumerate}

\end{document}


