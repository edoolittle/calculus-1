\documentclass[12pt]{article}
\usepackage{mathptmx}
\usepackage{fullpage}
\usepackage{multicol}
\usepackage{amsmath,amssymb}
\usepackage{graphicx}
\usepackage[aux]{rerunfilecheck}

\newcommand{\ds}{\displaystyle}

\reversemarginpar

\title{MATH 110-004 200730 Quiz 9 Solutions}
\author{Edward Doolittle}

\begin{document}
\maketitle

\begin{enumerate}
\item By the Fundamental Theorem of Calculus we have
  \begin{align*}
    \int_0^1 (1+x^3)^2 \; dx
    = \left. \int (1+x^3)^2 \; dx \right|_0^1
  \end{align*}
  Evaluating the indefinite integral,
  \begin{align*}
    \int (1+x^3)^2 \; dx
    = \int (1+2x^3+x^6) \; dx
    = x + 2\frac{x^4}{4} + \frac{x^7}{7} + C
    = x + \frac{x^4}{2} + \frac{x^7}{7} + C
  \end{align*}
  Evaluating the definite integral,
  \begin{align*}
    \int_0^1 (1+x^3)^2 \; dx
    = \left. x + \frac{x^4}{2} + \frac{x^7}{7} \right|_0^1
    = \left(1 + \frac{1}{2} + \frac{1}{7}\right) 
    - \left(0 + \frac{0}{2} + \frac{0}{7}\right)
    = 1 + \frac{1}{2} + \frac{1}{7}
  \end{align*}
\item We break the integral into two parts and evaluate each separately:
  \begin{align*}
    \int_{-1}^3 f(x) \; dx 
    = \int_{-1}^1 f(x) \; dx + \int_1^3 f(x) \; dx
  \end{align*}
  The first part is 
  \begin{align*}
    \int_{-1}^1 f(x) \; dx
    = \int_{-1}^1 (2-x) \; dx
    = \left. 2x-\frac{x^2}{2} \right|_{-1}^1
    = \left( 2(1)-\frac{1^2}{2}\right)
    - \left( 2(-1)-\frac{(-1)^2}{2}\right)
    = 4
  \end{align*}
  and the second part is
  \begin{align*}
    \int_1^3 f(x) \; dx
    = \int_1^3 2x^2-1 \; dx
    = \left. 2\frac{x^3}{3} - x \right|_1^3
    = \left( 2\frac{3^3}{3} - 3 \right)
    - \left( 2\frac{1^3}{3} - 1 \right)
    = 18 - 3 - \frac{2}{3} + 1
    = 15 \frac{1}{3}
  \end{align*}
\end{enumerate}

\end{document}


