\documentclass[12pt]{article}
\usepackage{mathptmx}
\usepackage{fullpage}
\usepackage{multicol}
\usepackage{amsmath,amssymb}
\usepackage{graphicx}
\usepackage[aux]{rerunfilecheck}

\newcommand{\ds}{\displaystyle}

\reversemarginpar

\title{MATH 110-003 200730 Quiz 10 Solutions}
\author{Edward Doolittle}

\begin{document}
\maketitle

\begin{enumerate}
\item \textbf{Solution 1}: 
  We first try to find the indefinite integral
  $\ds \int \frac{x}{\sqrt{1+4x}} \; dx$.  We make the substitution
  $u=1+4x$, $du = 4\; dx$, $dx = du/4$, $x=(u-1)/4$ to obtain
  \begin{align*}
    \int \frac{x}{\sqrt{1+4x}} \; dx
    &= \int \frac{(u-1)/4}{u^{1/2}} \; \frac{du}{4}
    = \frac{1}{16} \int (u^{1/2} - u^{-1/2}) \; du
    = \frac{1}{16} \left(\frac{u^{3/2}}{3/2} - \frac{u^{1/2}}{1/2} +C\right)
    \\
    &= \frac{1}{24} (1+4x)^{3/2} - \frac{1}{8}(1+4x)^{1/2} + C
  \end{align*}
  You should check the integration by differentiating the final expression
  above to obtain the integrand.  (You will need to factor the lowest power
  of $(1+4x)$ that occurs in the expression from each term.)  Now we
  can evaluate the definite integral using the Fundamental Theorem of Calculus:
  \begin{align*}
    \int_2^6 \frac{x}{\sqrt{1+4x}} \; dx
    = \left. \frac{1}{24} (1+4x)^{3/2} - \frac{1}{8}(1+4x)^{1/2} \right|_2^6
    = \frac{1}{24} 5^3 - \frac{1}{8} 5 - \frac{1}{24} 3^3 + \frac{1}{8} 3
  \end{align*}
  You can stop there for full marks, or you can continue with the arithmetic
  to obtain the final answer $23/6$.

  \textbf{Soluton 2:} You can cut down somewhat on the amount of calculation
  required, at the expense of the ability to check your integration by 
  differentiating, if you change the limits of integration at the same time
  as you make the substitution.  We use the same substitution
  $u=1+4x$, $du = 4\; dx$, $dx = du/4$, $x=(u-1)/4$, but now we also note
  that when $x=2$, $u=9$ and when $x=6$, $u=25$ to obtain
  \begin{align*}
    \int_2^6 \frac{x}{\sqrt{1+4x}} \; dx
    &= \int_9^{25} \frac{(u-1)/4}{u^{1/2}} \; \frac{du}{4}
    = \frac{1}{16} \int_9^{25} (u^{1/2}-u^{-1/2}) \; du
    = \frac{1}{16} \left. 
      \left( \frac{u^{3/2}}{3/2} - \frac{u^{1/2}}{1/2} \right)
    \right|_9^{25}
    \\
    &= \frac{1}{16} \frac{2}{3} 5^3 - \frac{1}{16} 2\cdot 5
    - \frac{1}{16} \frac{2}{3} 3^3 + \frac{1}{16} 2 \cdot 3
  \end{align*}
  Again, you can leave the answer as-is or continue with the arithmetic
  to obtain $23/6$.

  Note: other substitutions are possible; you may find that the
  substitution $u=(1+4x)^{1/2}$ works better, but it is harder to 
  guess.
\item Differentiating both sides of the equation,
  \begin{align*}
    \frac{d}{dx} \left( 6 + \int_a^x \frac{f(t)}{t^2} \; dt \right)
    = \frac{d}{dx} 2\sqrt{x}
  \end{align*}
  or, by the Fundamental Theorem of Calculus 1,
  \begin{align*}
    \frac{f(x)}{x^2} = 2 \frac{1}{2} x^{-1/2} = x^{-1/2}
  \end{align*}
  Solving for $f$ gives $f(x) = x^{3/2}$.

  Now letting $x=a$ we have
  \begin{align*}
    6 + \int_a^a \frac{f(t)}{t^2} \; dt = 2\sqrt{a}
  \end{align*}
  or, using the fact that $\int_a^a$ of anything is $0$,
  \begin{align*}
    6 = 2\sqrt{a} \implies a=9
  \end{align*}
  
  In summary, $f(x)=x^{3/2}$ and $a=9$ makes the equation in the 
  question true for all $x>0$.
\end{enumerate}

\end{document}


