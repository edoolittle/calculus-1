\documentclass[12pt]{article}
\usepackage{mathptmx}
\usepackage{fullpage}
\usepackage{multicol}
\usepackage{amsmath,amssymb}
\usepackage[aux]{rerunfilecheck}

\newcommand{\ds}{\displaystyle}

\reversemarginpar

\usepackage{lastpage,fancyhdr}
\usepackage{fancyhdr}
\pagestyle{fancy}
\lhead{MATH110 200930 Quiz 10      \\
  Time: 20 minutes                        \\ \quad }
\chead{Page\ \thepage\ of \pageref{LastPage}   \\ \quad \\ \quad}
\rhead{Name: \underline{\hspace{1.5in}}        \\
  Student \#: \underline{\hspace{1.5in}}  \\ \quad }
\cfoot{}
\addtolength{\headheight}{\baselineskip}
\addtolength{\headheight}{\baselineskip}
\addtolength{\headheight}{\baselineskip}
\addtolength{\headheight}{\baselineskip}
\renewcommand{\headrulewidth}{0pt}
\fancypagestyle{plain}{%
  \lhead{}
  \chead{FIRST NATIONS UNIVERSITY OF CANADA                \\
    DEPARTMENT OF SCIENCE \\
    MATH110 200930 Quiz 10     \\
    \quad                                      }
  \rhead{}
  \cfoot{Page\ \thepage\ of \pageref{LastPage}}
}

\begin{document}
\thispagestyle{plain}

\begin{flushleft}
Time:  20 minutes                \hfill       Name: \underline{\hspace{2in}} \\
Instructor: Dr. Edward Doolittle \hfill Student \#: \underline{\hspace{2in}}
\end{flushleft}

\noindent
Please\marginpar{\centering (marks)} do questions 1 and 2.  You have 10 minutes
to do each question, for a total of 20
minutes for the quiz.  A 
calculator of the type mentioned in the course outline is allowed.
%but is not necessary.  
%You may leave early if you can
%do so without disturbing any of your colleagues.
%If you finish early, I suggest you check your work thoroughly.
%\textbf{Please do not disturb your colleagues by climbing over them while
%they are trying to write the quiz.}

\begin{enumerate}
\item Express\marginpar{\centering (10)} 
  the limit
  $\ds\lim_{n\to\infty}\sum_{i=1}^n
    \left(2x_i^*-5(x_i^*)^2+\sin(x_i^*)\right) \Delta x$ 
  as a definite integral on the interval $[0,\pi]$.  
  Do not attempt to evaluate the integral.
\vfill
\newpage
\item Evaluate\marginpar{\centering (10)} 
  the definite integral $\ds \int_0^1 (x^3+2) dx$ from first principles,
  i.e., from the definition of a definite integral.  You may find some of
  the following formulas helpful.
  \begin{align*}
    \sum_{i=1}^n i = \frac{n(n+1)}{2}
    \qquad
    \sum_{i=1}^n i^2 = \frac{n(n+1)(2n+1)}{6}
    \qquad
    \sum_{i=1}^n i^3 = \left(\frac{n(n+1)}{2}\right)^2
  \end{align*}
\vfill
\end{enumerate}

\end{document}

